

\subsection{Cobb-Douglas}
\label{sec:cobb-douglas}

\begin{definition}[Cobb-Douglas]
  The $n$-good Cobb-Douglas utility function is
  \[
  u(x) = A \prod_{i=1}^N x_i^{\alpha_i}
  \]
  where $A > 0$ and $\sum_i \alpha_i = 1$.
\end{definition}

\subsubsection{Deriving Marhsallian Demand}

Note first that $\log $ is monotone, so maximizing $\log u$ is
equivalent to maximizing $u$. Hence our UMP is
\[
\max_{\{x_i\}} \sum \alpha_i \log x_i \quad s.t. \quad \sum_i p_i x_i = w
\]
Note: the budget constraint holds with strict equality since $u$ is
increasing. Moreover, since $\log 0 = - \infty$, we will have an
interior solution -- nonzero consumption of $x_i \; \forall i$.

Our Lagrangian for the constrainted maximization is
\[
{\cal L} = \sum \alpha_i \log x_i - \lambda(\sum_i p_i x_i - w)
\]
which gives the FOCs
\[
\frac{\partial {\cal L}}{\partial x_i}
= \frac{\alpha_i}{x_i} - \lambda p_i
= 0
\]
Thus we have, $\forall i, k$
\[
p_i x_i = \frac{\alpha_i}{\alpha_k} p_k x_k
\]
which we can substitute into our budget constraint
\begin{align}
  \sum_i p_i x_i & = w & \notag \\
  \sum_i \frac{\alpha_i}{\alpha_k} p_k x_k & = w
  & \text{substituting from above} \notag \\
  \frac{p_kx_k}{\alpha_k} \sum \alpha_i & = w & \notag \\
  \frac{p_kx_k}{\alpha_k} & = w 
  & \sum_i \alpha_i = 1 \; \text{by assumption} \notag \\
  x_k(p, w) & = \frac{\alpha_k w}{p_k} & \notag
\end{align}
which is true for each of our goods. It is immediately clear that this
is homogenous of degree $1$ in both prices and income. Additionally,
we can quickly verify that the income elasticity of demand is $1$:
\[
\frac{\partial x_j(p,w)}{\partial w} \frac{w}{x_k(p,w)}
= \frac{\alpha_k}{p_k} \frac{w}{\frac{\alpha_k w}{p_k}}
= 1
\]

\subsubsection{Deriving Indirect Utility}

We obtain the indirect utility function by simple substituting the
Marshallian demand into the direct utility function, i.e.
\begin{align}
  v(x(p, w)) & = \prod_i x_i^{\alpha_i} & \notag \\
  & = \prod_i \left(\frac{\alpha_i w}{p_i}\right)^w & \notag \\
  & = \prod_i w^{\alpha_i} \left(\frac{\alpha_i}{p_i}\right)^{\alpha_i} & \notag \\
  & = w^{\sum_i \alpha_i} \prod_i \left(\frac{\alpha_i}{p_i}\right)^{\alpha_i} & \notag \\
  & = w \prod_i \left(\frac{\alpha_i}{p_i}\right)^{\alpha_i} & \sum_i \alpha_i = 1 \notag 
\end{align}


\subsubsection{Computing Expenditure Function}

Consider a fixed price vector $p$. And let 
\[
C := \prod_i \left(\frac{\alpha_i}{p_i}\right)^{\alpha_i}
\]
which is a constant at prevailing prices. Then the utility I get at
each level of expenditure/wealth is
\[
v(w) = Cw
\]
hence the minimum expenditure $w$ to get $u$ utility is $\frac{u}{C}$. 
Hence 
\begin{align}
  e(p,u) & = \frac{u}{C} & \notag \\
  & = u \prod_i \left(\frac{p_i}{\alpha_i}\right)^{\alpha_i} & \notag
\end{align}



\subsubsection{Computing Hicksian Demand}

We know that $h(p, u) = x(p, e(p,u))$. Hence
\begin{align}
  h_i(p,u) & = x_i(p, e(p,u)) & \notag \\
  & = \frac{\alpha_i e(p, u)}{p_i} & \notag \\
  & = \frac{\alpha_i}{p_i} \cdot
  u \prod_i \left(\frac{p_i}{\alpha_i}\right)^{\alpha_i} & \notag
\end{align}

\subsection{Leontief}
\label{sec:leontif}

\begin{definition}[Leontief Preferences]
  Leontief preferences are given by
  \[
  x'' \succeq x' \iff \min\{x_i''\} \geq \min\{x_i'\}
  \]
\end{definition}

Note: Leontief preferences are a case of preferences that are
representable by a continuous utility function, but no differentiable
utility function can represent them. The issue is at the kink where
$x_i = x_j$ for all components. In the case of two components, the
indifference curves are $L$ shaped. Note that Leontief preferences
\textit{are} quasiconcave, but are not strictly quasiconcave.


\subsection{CES}
\label{sec:ces}

\begin{definition}[CES Utility]
  The CES (Constant Elasticity of Substitution) utility function is given by
  \[
  u(x) = \sum_{i=1}^n \alpha_i x_i^\gamma
  \]
  with $\gamma \in [0,1]$ and $\alpha_i > 0$.
\end{definition}

\subsubsection{Deriving the Marshallian Demand}

Now the consumer's problem is 
\[
\max_{\{x_i\}} \sum \alpha_i x_i^\gamma \quad s.t. \quad \sum_i p_i x_i = w
\]
which gives the Lagrangian
\[
{\cal L} = \sum \alpha_i x_i^\gamma - \lambda\left[\sum_i p_ix_i - w\right]
\]
(since utility is increasing in $x$ we have strict equality on the
constraint).

We can differentiate to get the FOC on each $x_i$
\[
\gamma\alpha_ix_i^{\gamma-1} - \lambda p_i = 0
\implies \lambda = \frac{\gamma \alpha_i}{p_i} x_i^{\gamma-1}
\]
hence
\begin{align}
  \lambda & = \lambda & \notag \\
  \frac{\gamma \alpha_i}{p_i} x_i^{\gamma-1} 
  & = \frac{\gamma \alpha_k}{p_k} x_k^{\gamma-1} & \notag \\
  x_i^{\gamma-1} &= 
  \left(\frac{\alpha_i}{p_i}\right)^{-1}
  \left(\frac{\alpha_k}{p_k}\right)
  x_k^{\gamma-1} & \notag \\
  x_i & = 
  \left(\frac{\alpha_i}{p_i}\right)^{\frac{-1}{\gamma-1}}
  \left(\frac{\alpha_k}{p_k}\right)^{\frac{1}{\gamma-1}}
  x_k & \notag
\end{align}
which we can substitute into our budget constraint
\begin{align}
  \sum_i p_i x_i & = w & \notag \\
  \sum_i p_i \left(   
    \left(\frac{\alpha_i}{p_i}\right)^{\frac{-1}{\gamma-1}}
    \left(\frac{\alpha_k}{p_k}\right)^{\frac{1}{\gamma-1}}
    x_k 
  \right) & = w & \text{substituting from above} \notag \\
  x_k \left(\frac{\alpha_k}{p_k}\right)^{\frac{1}{\gamma-1}}
  \sum_i p_i \left(\frac{\alpha_i}{p_i}\right)^{\frac{-1}{\gamma-1}} & = w & \notag \\
  x_k(p,w) & 
  = \frac{
    \left(\frac{p_k}{\alpha_k}\right)^{\frac{1}{\gamma-1}}
  } {
    \sum_i p_i \left(\frac{p_i}{\alpha_i}\right)^{\frac{1}{\gamma-1}}
  } & \notag
\end{align}
which gives our Marshallian demand.

Note that the general trick here is that we set the Lagrange
multipliers equal to themselves (we only have a single constraint),
then then solve for a general $x_i$ in terms of some fixed $x_k$. We
then substitute this into our budget constraint to get $x_k$ only as a
function of prices, wealth, and the parameters of the utility function.


For more information, or for further functional forms analysis, see
the file ``Consumer\_Theory\_Functional\_Forms.pdf'' in the parent
directory of these notes. These are not, however, required for the
exam.
