\message{ !name(monotone.tex)}
\message{ !name(monotone.tex) !offset(-2) }


\subsection{Motivation}
\label{sec:motivation}

We are now concerned with comparative statics, that is, given some
maximization problem
\[
\max_{x} f(x; \theta)
\]
we want to know both how the maximum value ($v$) and the maximizing
argument ($x$) depend upon $\theta$. Assuming all derivatives exist
and the solution is interior and unique (which we could get, e.g. if
$f$ is ${\cal C}^2$, and strictly convex), we can apply the implicit
function theorem since the maximizer uniquely solves
\[
f_x(x(\theta), \theta) = 0
\]
and we can say locally that
\[
\frac{\partial x(\theta)}{\partial \theta}
= - \frac{f_{x\theta}(x(\theta), \theta)}{f_{xx}(x(\theta),\theta)}
\]
but this breaks down if any of the above assumptions fails
(e.g. differentiability, convexity), and we want our analysis to be
robust to specifications that might give a different $f$. This
introduces the idea of ``Robust Comparative Statics'' which seek to
avoid calculus and find conditions which can be used even if functions
are nondifferentiable, discontinuous, or even defined on discrete
state spaces.


\subsection{Preliminaries}
\label{sec:preliminaries}

\begin{definition}[Partial Order]
  A relation $R$ that satisfies transitivity, reflexivity, and
  antisymmetry is called a partial order.
\end{definition}

\begin{definition}[Partially Ordered Set]
  A partially ordered set, or poset, is a partial order relation
  together with a set, $(X, R)$. Note that a poset does not require
  completeness.
\end{definition}

\begin{definition}[Meet]
  The meet operator is defined as follows:
  \[
  x \wedge y = \max\{z | xRz, yRz\}
  \]
\end{definition}

\begin{definition}[Join]
  The join operator is defined as follows:
  \[
  x \vee y = \min\{z | zRx, zRy\}
  \]
\end{definition}

\begin{definition}[Lattice]
  A lattice is a poset that is closed under meet and join operations.
\end{definition}

\begin{definition}[Strong Set Order]
  The strong set order is given by the binary relation
  \[
  A \succeq_S B \iff
  a \vee b \in A, \quad
  a \wedge b \in B, \quad
  \forall a \in A, \forall b \in B
  \]
\end{definition}

\begin{definition}[Increasing Differences]
  Let $f$ be a function $f : X \times \Theta \to \R$ where both $X$
  and $\Theta$ are lattices. Consider $x^H, x^L \in X$, and $\theta^H,
  \theta^L \in \Theta$ with $x^H \geq_X x^L$ and $\theta^H \geq_\Theta
  \theta^L$. Then $f$ has increasing differences if
  \[
  f(x^H, \theta^H) - f(x^L, \theta^H) \geq
  f(x^H, \theta^L) - f(x^L, \theta^L)
  \]
  this is the same as saying that the difference $f(x^H, \theta) -
  f(x^L, \theta)$ is weakly increasing in $\theta$.
\end{definition}

\begin{definition}[Supermodularity]
  Let $f$ be a function $f : X \to \R$ where $X$ is a lattice. Then
  $f$ is supermodular if
  \[
  f(x^H \vee x^L) + f(x^H \wedge x^L) \geq f(x^H) + f(x^L) \quad
  \forall x^H, x^L \in X
  \]
\end{definition}

\begin{definition}[Submodularity]
  A function $f$ is submodular if $-f$ is supermodular.
\end{definition}

\begin{prop}
  A differentiable function $f$ is supermodular if and only if
  \[
  \frac{\partial^2 f}{\partial x_i \partial x_j} \geq 0
  \quad \forall i \neq j
  \]

 Similarly, a differentiable function $f$ is submodular if and only if
  \[
  \frac{\partial^2 f}{\partial x_i \partial x_j} \leq 0
  \quad \forall i \neq j
  \]
\end{prop}

\begin{definition}[Substitutes]
  Inputs $x_i$ and $x_j$ to an objective function $f$ are substitutes
  at input vector $x$ if $f$ is submodular in these arguments when
  evaluated locally around $x$ (i.e. all other inputs are fixed at
  their levels in $x$). In the special case where $f$ is
  differentiable, this is captured by
  \[
  \frac{\partial^2 f}{\partial x_i \partial x_j} \leq 0
  \]
\end{definition}

The intuition behind this definition of substitutes is that
submodularity implies decreasing differences, so for a fixed level of
output, I could slightly decrease one input, then increase the other
input, and then the amount by which I would have to increase the first
input in order to meet the fixed level of output would be lower than
the amount by which I decreased it initially, hence I can get away
with a lower absolute level of the first input by increasing the level
of the second input, which is the very idea of substitutes.


** TODO see the proofs we did regarding supermodularity on the problem
set! Those are especially useful here. **

\subsection{Topkis}
\label{sec:topkis}

\begin{theorem}[Topkis Monotonicity Theorem]
  Consider an optimization problem of the form
  \[
  \max_{x} f(x, \theta)
  \]
  and let $x^*(\theta)$ be the maximizer correspondence. If $f(x,
  \theta)$ has increasing differences, then $x^*(\theta)$ is ewakly
  increasing in $\theta$.
\end{theorem}

\begin{prop}
  Define
  \[
  x^{**}(\theta) = \sup\{ \arg \max_x f(x,\theta) + g(x) \}
  \]
  Then $x^{**}(\theta)$ is weakly increasing in $\theta$ \textbf{for
    all $g$} if and only if $f$ has increasing differences.
\end{prop}

\begin{theorem}[Topkis]
  Consider an optimization problem of the form
  \[
  \max_{x \in D} f(x, \theta)
  \]
  where $D$ is a lattice (with order $\succeq_X$) and does not depend
  on $\theta$. If $f$ exhibits increasing diferences in $(x,\theta)$
  relative to $(\succeq_X, \succeq_\Theta)$ and is supermodular in $x$
  (relative to $\succeq_X$), then the optimal choice correspondence
  $x^*(\theta)$ will be increasing in the strong set order relative to
  $\succeq_X$ (where an increase in the parameter is relative to
  $\succeq_\Theta$).

  Under the above conditions, we can conclude that $\forall \theta,
  \theta' \in \Theta$ with $\theta' \succeq_\Theta \theta$ and
  $\forall x \in x^*(\theta), x' \in x^*(\theta')$,
  \[
  x \wedge x' \in x^*(\theta) \quad \text{and} \quad x \vee x' \in x^*(\theta')
  \]
\end{theorem}


\subsection{Milgrom-Shannon}
\label{sec:milgrom-shannon}

\begin{definition}[Single Crossing]
  A function $f: \R^2 \to \R$ is single crossing in $(x, \theta)$ if
  $\forall x' > x$ and $\theta' > \theta$,
  \begin{enumerate}[(i)]
  \item $f(x', \theta) - f(x, \theta) \geq 0 \implies f(x', \theta') -
    f(x, \theta') \geq 0$
  \item $f(x', \theta) - f(x, \theta) > 0 \implies f(x', \theta') -
    f(x, \theta') > 0$
  \end{enumerate}

  We say the single crossing is strict if $f(x', \theta) - f(x,
  \theta) \geq 0 \implies f(x', \theta') - f(x, \theta') > 0$.
\end{definition}

Note that unlike increasing differences, this condition is not
symmetric in the variables $(x, \theta)$.

\begin{prop}
  If $f$ has increasing differences then it is also single crossing,
  but the converse is not true.
\end{prop}

\begin{prop}[Milgrom-Shannon Monotonic Selection Theorem]
  Define
  \[
  x^*(\theta, \bar X) := \arg \max_{x \in X} f(x, \theta)
  \]
  If $f$ is single-crossing in $(x, \theta)$, then $x^*(\theta, \bar
  X)$ is weakly increasing in $\theta \; \forall \bar X \subset X$.

  Moreover, if $x^*(\theta, \bar X)$ is non-decreasing in $\theta$ for
  all finite sets $\bar X \subset \R$, then $f$ is single-crossing in
  $(x, \theta)$.
\end{prop}

The usefulness of the Monotonic Selection theorem is that we can make
a statement about comparative statics that is robust to arbitrarily
selecting/restricting the feasible set $\bar X \subset X$.

\subsection{Applications in Producer Theory}
\label{sec:appl-prod-theory}

\begin{prop}
  Suppose a firm faces the cost minimization problem
  \[
  \max_{k \in K} -rk - wl \quad s.t. \quad F(k,l) \geq q
  \]
  then the optimal choice of capitak $k^*(r,w,q)$ is weakly decreasing
  in $r$.
\end{prop}

\begin{prop}
  Now suppose instead of prices, the firm is adjusting its capital in
  response output $q$, but now required labor is a function of $k$ and
  $q$
  \[
  \max_k -rk - wL(k,q)
  \]
  and $k^*(r,w,q)$ is weakly increasing in $q$ if $-L(k,q)$ has
  increasing differences.
\end{prop}

This proposition is difficult to interpret. Note the $L$ is the
function such that $F(k, L(k,q)) = q$. The core idea is that starting
from a fixed level of capital, the amount of labor required to
substitute for a small decreas in capital while holding output fixed
increases as the desired level of output increases.

Now we turn our attention to a much more interesting problem: cross
price effects in the firm's profit maximization problem. In
particular, we want to know what happens to $k$ as $w$ changes.

\begin{definition}[Substitutes and Complements]
  $k$ and $l$ are substitutes in production if $F(k,l)$ has decreasing
  differences. $k$ and $l$ are complements in production if $F(k,l)$
  has increasing differences.
\end{definition}

We now turn to a few common examples. In the case of twice
continuously differentiable functions, we can evaluate the second
derivative and check the sign directly.





\subsection{Long Run versus Short Run Response}
\label{sec:LR-SR}

\subsubsection{Rough Idea}

There are an interesting set of economic models where the long-run
responses to price changes are larger than short run responses. The
key idea is some sort of ``positive feedback'' argument.

First, we start with some production function $f(k, l)$ where inputs
are substitutes. We then assume taht in the short run $k$ is fixed, so
when wages increase labor falls in the short run (and so does output),
but capital is fixed. In the long run, since the inputs are
substitutes and before the wage change I was indifferent between an
extra unit of $k$ versus $l$, now with the wage change I prefer the
additional unit of $k$ so I will tend to increase $k$ in the long
run. But since they are substitutes, the long run increase in $k$
lowers the marginal product of labor, so $l$ further decreases in the
long run. Hence the LR effect is larger than SR due to capital
flexibility. Note that if $k$ and $l$ were complements the opposite
effect would take place, but still LR effects are greater than in SR.


\subsubsection{Formalization}



Samuelson suggested that a firm would react more to input price
changes in the long-run than in the short-run because it has more
inputs it can adjust. We take the ``short run'' to be the period when
$k$ is effectively fixed, whereas the ``long run'' allows $k$ to
vary. We can be more precise about this relationship, though.

\begin{prop}[Milgrom and Roberts]
  Suppose $X, Y = \R$ and define
  \[
  x(y, t) := \arg \max_{x \in X} F(x,y,t)
  \]
  and
  \[
  y(t) := \arg \max_{y \in Y} F(x(y, t), y, t)
  \]
  Suppose further that $F: X \times Y \times \R \to \R$ is
  supermodular, that $t, t' \in \R $ with $t' \geq t$, and that the
  maximizes described below are unique for parameter values $t$ and
  $t'$. Then
  \[
  x(y(t'), t') \geq x(y(t), t') \geq x(y(t), t)
  \]
  and
  \[
  x(y(t'), t') \geq x(y(t'), t) \geq x(y(t), t)
  \]
\end{prop}

The above may be useful to characterize long run response. For
instance, if $x$ and $y$ happen to be complements, then the objective
would be supermodular in $x$ and $y$ and the above would characterize
growth paths.

\begin{prop}[LeChatelier Principle]
  If production is $f(k,l)$ and wage $w > w_0$ increases, and if $k$
  and $l$ are always complements or always substitutes, then
  \[
  l_{LR}(w) \leq l_{SR}(w_0, w) \leq l_{LR}(w_0)
  \]
\end{prop}

Note that when we have always complements or always substitutes, we
get supermodularity, and we can apply Milgrom-Roberts, which gives the
result.

Note also that the above does not apply when we have sometimes complements
and sometimes substitutes, which would likely be the case in many
applications. For instance, perhaps capital is a complement to labor
when labor is abundant but a substitute when labor is scarce.

\message{ !name(monotone.tex) !offset(-351) }
