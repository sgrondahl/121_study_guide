

\subsection{Utility Functional Forms}
\label{sec:util-funct-forms}

\begin{definition}[Quasi-Concave]
  A function $f : X \to \R$ is (strictly) quasiconcave if every upper level set
  of $f$ is convex, i.e. $\forall a \in \R$,
  \[
  P_a := \{x \in S \; | \: f(x) \geq a\}
  \]
  is (strictly) convex.
\end{definition}

\begin{prop}
  The function $f$ is quasiconcave if and only if $\forall x \in S,\;
  \forall x' \in S, \; \forall \lambda \in [0,1]$,
  \[
  f(x) \geq f(x') \implies f((1-\lambda)x + \lambda x') \geq f(x')
  \]
  that is, the line segment connecting any two level curves lies
  nowhere below the level curve corresponding to the lower value of
  the objective. The function is strictly quasiconcave if the
  inequality is strict.
\end{prop}

\begin{prop}
  A concave function is quasiconcave. A convex function is
  quasiconvex. (The converse is not necessarily true).
\end{prop}

\begin{prop}
  If $f$ is quasiconcave and $g: \R \to \R$ is increasing, then $g
  \circ f$ is quasiconcave.
\end{prop}


\textbf{Why do we care about quasiconcavity?} Ah, good question! The
reason is that we want every upper level set of the utility function
(i.e. the set bounded below by an \textit{indifference curve}) to be
convex. If we did not have this condition, then in general we would
not get interior solutions. Consider quasiconvex indifference curves
(i.e. level curves of the utility function). Then we would
\textit{always} get a corner solution for optimal utility under a
linear budget constraint.


For completeness, we include definitions also of quasiconvexity,
though we tend not to use these at all.

\begin{definition}[Quasi-Convex]
  A function $f : X \to \R$ is quasiconvex if every lower level set
  of $f$ is convex, i.e. $\forall a \in \R$,
  \[
  P_a := \{x \in S \; | \: f(x) \leq a\}
  \]
  is convex.
\end{definition}

\begin{prop}
  If $f$ is quasiconvex and $g: \R \to \R$ is decreasing, then $g
  \circ f$ is quasiconvex.
\end{prop}

\begin{definition}[Homogenous Function]
  A function $f: V \to W$ between vector spaces over a field $F$ if
  homogenous of degree $k$ ($k \in \Z_+$) if
  \[
  f(\alpha v) = \alpha^kf(v)
  \]
  $\forall \alpha \in F, \alpha \neq 0, v \in V$.
\end{definition}

\begin{definition}[Homothetic Preferences]
  A set of preferences is homothetic if for any bundles $(x,y)$ and
  $(x', y')$, and any $\alpha > 0$,
  \[
  (x,y) \sim (x', y') \implies (\alpha x, \alpha y) \sim (\alpha x', \alpha y')
  \]
\end{definition}

\begin{prop}[Homothetic Utility]
  Preferences are homothetic if and only if they admit representation
  by a utility function with the following property:
  \[
  u(x, y) = u(x', y') \implies u(\alpha x, \alpha y) = u(\alpha x', \alpha y')
  \]
  $\forall \alpha > 0$. 
\end{prop}

\begin{prop}
  Homothetic preferences admit representation by a utility function
  with the following property:
  \[
  u(\alpha x, \alpha y) = \alpha u(x,y)
  \]
  $\forall \alpha > 0$. Note that unlike the previous result, this is
  not bidirectional; many utility functions without the above property
  represent homothetic preferences.
\end{prop}

\begin{proof}
  We proceed by construction. Construct $u$ such that $u(z,z) =
  z$. Now consider any point $(x,y)$ and find the $(z,z)$ such that
  $(x,y)\sim(z,z)$ and note then that $u(x,y) = z$. Then it must be
  the case that $(\alpha x, \alpha y) \sim (\alpha z, \alpha
  z)$. Hence $u(\alpha x, \alpha y) = \alpha z = \alpha u(z, z) =
  \alpha u( x,y)$, competing the proof.
\end{proof}

Note: Cobb-Douglas and Leonteif utility functions are homothetic.

\subsection{Walrasian Equilibrium Existence Preliminaries}
\label{sec:walr-equil-exist}

Note: Parag put these in lecture 8 but never used them to prove
existence.

\begin{theorem}[Separating Hyperplane Theorem]
  If $B, C \subseteq \R^L$ are convex and nonempty, and $B \cap C =
  \emptyset$, then $\exists p \in \R^L, p \neq 0$ such that
  \[
  \sup_{b \in B} p \cdot b \leq \inf_{c \in C} p \cdot c
  \]
\end{theorem}

\begin{theorem}[Supporting Hyperplane Theorem]
  Suppose that $B \subseteq \R^L$ is conved and that $x \not\in
  \mathring B$. Then $\exists p \in \R^L, p \neq 0$ such that 
  \[
  p \cdot y \geq p \cdot y \quad \forall y \in B
  \]
\end{theorem}


\subsection{Implicit Function Theorem}
\label{sec:impl-funct-theor}

\begin{definition}[Implicit Function Theorem]
Let $W$ be an open neighborhood of $c \in \R^{n+m}$ and $F:W \to \R^N$
be differentiable function with $F(c) = 0$ and $[DF(c)]$ onto (hence
it has rank $n$). Arrange $[DF(c)]$ such that $n$ linearly independent
columns come first (left), and similarly rearrange $c$ such that the
entries correspond to the pivotal unknowns for the first $n$ entries
followed by the next $m$ entries. Then there exists a unique
continuously differentiable mapping $g$ from a neighborhood of $b$ to
a neighborhood of $a$ that uniquely specifies the $n$ pivotal
components as a function of the $m$ nonpivotal components. That is,
\[
g(b) = a \quad and \quad F(g(y), y) = 0 \quad \forall y \in B_r(b)
\]
and the derivative is given by
\[
[Dg(b)] = -[D_1F(c), \dots, D_nF(c)]^{-1}[D_{n+1}F(c), \dots, D_{n+m}F(c)]
\]
  
\end{definition}
