

Recall that SWT requires such great information to construct
supporting prices, that we question why given so much information why
we wouldn't just control the economy directy (why do we need prices?).
Market mechanism works by using prices to give people concise
sufficient statistics allowing them to make coordinated choices and
arrive at a socially optimal allocation.

The general approach regarding prices is to formalize the statement
that it is impossible to have a mechanism verifying a Pareto optimal
allocation without revealing a supporting Walrasian equilibrium (or
common marginal rates of substitution, which gives supporting prices)
This provides a formalization of the necessity of price revelation for
informational reasons.


\subsection{Microfoundations and the Core}
\label{sec:micr-core}

\begin{definition}[Core]
  The core is the set of all allocations such that no coalition (set
  of agents) can improve on or block the allocation (make all of its
  members better off) by seceding from the economy and only trading
  among its members.
\end{definition}

Note that the core is an institution free concept, and does not
require prices.

Our central result is that for economies with a large number of
agents, core allocations are ``approximately Walrasian.'' This gives a
positive foundation for price taking. Core convergence and
nonconvergence allows us to identify situations in which price-taking
is more or less reasonable.


\begin{definition}[Coalition]
  In an exchange economy, a coalition is a set $S \subset I$ of agents.
\end{definition}

\begin{definition}[Blocking Coalition]
  A coalition $S$ blocks/improves on an allocation $x$ by $\hat x$ if 
  \[
  \forall i \in S, \; \hat x^i \succeq_i x^i
  \]
  with at least one preference strict, subject to the constraint
  \[
  \sum_{i \in S} \hat x^i = \sum_{i \in S} \omega^i
  \]
\end{definition}

\begin{definition}[Core]
  The core (by weak domination) is the set of all allocations which
  cannot be blocked/improved on by any coalition.
\end{definition}

\begin{prop}
  In an exchange economy, every core allocation is Pareto optimal.
\end{prop}

\begin{prop}
  In an exchange economy, every Walrasian Equilibrium lies in the
  core.
\end{prop}

\begin{proof}
  Let $(x^*, p^*) \in W({\cal E})$ and suppose $(x^*, p^*) \not\in
  C({\cal E})$. The there exists coalition $S \subset I$ such that for
  all $i \in S$, $x^{i'} \succeq^i \omega^i$ (with one inequality
  strict). Hence it must be the case that these new allocations are
  unaffordable at $p^*$, else they would be the Walrasian equilibrium,
  hence $p^* \cdot x^{i'} \geq p^* \cdot \omega^i$ with the inequality
  strict for one guy. We can sum over the guys in the coalition to get
  \[
  \sum^i p^* \cdot x^{i'} > \sum^i p^* \cdot \omega^i
  \]
  hence for some good $l$
  \[
  \sum_i p^*_l \cdot x^{i'}_l > \sum_i p^*_l \omega^i_l
  \]
  and since prices are nonnegative, it must be the case that
  \[
  \sum_i x^{i'}_l > \sum_i \omega^i_l
  \]
  which violates the resource constraint on the coalition, completing
  the proof.
\end{proof}

Note that this adds strength to the FWT, since it says that no group
(not just individual) can be made better off!

Recall that in the two consumer Edgeworth box, every allocation in the
contract curve is in the core, but only one is a Walrasian
allocation. However, as we increase the size of the economy, the
non-Walrasian allocations gradually drop from the core until in the
limit only the Walrasian allocations are left.

\subsection{Core Convergence}
\label{sec:core-convergence}

\begin{definition}[Type]
  The set of types of consumers is ${\cal I} = \{1, \dots, I\}$.
\end{definition}

\begin{definition}[R-Replica Economy]
  The R-replica economy ${\cal E}_R$ is one with $R$ consumers ofeach
  type, hence the tptal number of consumers is $R \cdot I$.
\end{definition}

\begin{definition}[Allocation]
  An allocation in the R-replica economy is a set $(x^{i,r}) \in
  R^{LRI}_+$, where $x^{i,r}$ is the bundle of the $r^{th}$ consumer
  of type $i$. The budget constraint in this case is
  \[
  \sum_{i \in {\cal I}} \sum_{r=1}^R x^{i,r} = R \sum_{i \in {\cal I}} \omega^i
  \]
\end{definition}


\begin{prop}[Equal Treatment]
  Suppose consumers have strictly convex preferences. If $x$ is a core
  allocation in ${\cal E}_R$, then $x^{i,r} = x^{i,r'} \; \forall r,
  r'$ (i.e. consumers of the same type consume the same bundle).
\end{prop}

Hence we can regard core allocations as vectors of fixed size $LI$,
irrespective of the replica we are concerned with.

\begin{definition}[Type Allocation]
  The type allocation for any replica $R$ is given by
  \[
  (x_1, \dots, x_{I}) \in \R^{LI}_+
  \]
  where $x_i$ is the equal treatment allocation for each consumer of
  type $i$.
\end{definition}

\subsubsection{Constructing a Walrasian Equilibrium from the Core}
\label{sec:constr-walr-equil}

\begin{theorem}[Debreu-Scarf 1963]
  Let $C_R \subset \R^{LI}_+$ denote the set of type allocations
  corresponding to the equal treatment core allocations in the
  $R$-replica economy. Then
  \[
  \cap_R C_R = W
  \]
  i.e. if $x^* \in C_R \; \forall R$, then $x^*$ is Walrasian.
\end{theorem}

\begin{proof}
  Let $C_R$ be as above, and let $W_R \subset \R^{LI}_+$ denote the
  type allocations corresponding to the equal treatment Walrasian
  allocations of the $R$-replica economy.

  \textbf{Observation 1.} Suppose coalition $S$ has an objection
  against a type allocation $x$ in the $R$-replica. Then the same is
  true in the $R+1$ replica, which implies
  \[
  C_{R+1} \subseteq C_R \; \forall R
  \]

  \textbf{Observation 2.} If a type coalition $x$ is Walrasian in the
  $R$-replica, it is also $Walrasian$ in the $R+1$ replica and vice
  versa:
  \[
  W_R = W_{R+1} = W \; \forall R
  \]

  \textbf{Observation 3.} By our earlier proposition
  \[
  W \subset C_R \; \forall R
  \]

  In combination, these observations generate
  \[
  W \subseteq \dots \subseteq C_{R+1} \subseteq C_R \subseteq \dots
  \subseteq C_1
  \]
\end{proof}
