

\begin{definition}[EMP]
  The consumer's expenditure minimization problem (EMP) is
  \[
  \min_{x \in X} p \cdot x \quad s.t. \quad u(x) \geq u_0
  \]
\end{definition}

\begin{prop}[UMP-EMP Duality]
  Suppose $u(x)$ is a continuous utility function satisfying local
  nonsatiation and $p >> 0$, then
  \begin{enumerate}[(a)]
  \item If $x^*$ is a solution to the UMP with wealth $w$, then $x^*$
    is a solution to the EMP with utility $u(x^*)$
  \item if $x^*$ is a solution to the EMP with utility $u_0$, then
    $x^*$ is a solution to the UMP with wealth $p \cdot x^*$.
  \end{enumerate}
\end{prop}

\begin{definition}[Expenditure Function]
  The expenditure function $e : X \times \R \to \R$ is defined by
  \[
  e(p, u) = \min_{x \in X: u(x) \geq u} p \cdot x
  \]
\end{definition}

\begin{definition}[Hicksian Demand]
  The Hicksian demand correspodnence $h : X \times \R \to X$ is
  defined by 
  \[
  h(p,u) = \{ z \in X | p \cdot z = e(p,u) \text{and} u(z) \geq u \}
  \]
  which yields the FOCs
  \[
  p_i  = \lambda \frac{\partial u(h(p,u))}{\partial x_i} 
  \quad \text{if} \quad  h_i(p,u) > 0
  \]
  and
  \[
  p_i  \geq \lambda \frac{\partial u(h(p,u))}{\partial x_i} 
  \quad \text{if} \quad h_i(p,u) = 0
  \]
\end{definition}

Note that $e(p,u) = p \cdot z \quad \forall z \in h(p,u)$. The
Hicksian demand is often called compensated demand since
\[
h(p,u) = x(p, e(p,u))
\]
that is, the Hicksian answers, ``how would demand change if we changed
prices and also gave wealth compensation so utility level is
unchanged?''


\begin{prop}
  Suppose $u$ is a continuous utility function satisfying local
  nonsatiation. Then the expenditure function $e(p,u)$ is
  \begin{enumerate}[(a)]
  \item Homogenous of degree $1$ in $p$, i.e. $e(\alpha p, u) = \alpha
    e(p,u)$
  \item strictly increasing in $u$ and nondecreasing in each $p_i$
  \item concave in $p$
  \item continuous in $p$ and $u$
  \end{enumerate}
\end{prop}

\begin{prop}
  Suppose $u$ is a continuous utility function satisfying local
  nonsatiation. Then the Hicksian demand correspondence $h(p,u)$ is
  \begin{enumerate}[(a)]
  \item Homogenous of degree $0$ in $p$, i.e. $h(\alpha p, u) = h(p,
    u), \forall \alpha, p, u$
  \item no excess utility property: $u(x) = u \forall x \in h(p,u)$
  \item If $u$ is strictly quasi-concave, then $h(p,u)$ is a function
  \end{enumerate}
\end{prop}

\begin{prop}[Compensated Law of Demand]
  Suppose $u$ is a continuous, strictly quasiconcave utility function
  satisfying local nonsatiation, then $\forall p', p''$,
  \[
  (p'' - p') \cdot [h(p'', u_0) - h(p', u_0)] \leq 0
  \]
  that is, the weighted average of price and demand movements (the dot
  product) is negative, i.e. demand goes down for products that have
  gone up in price.
\end{prop}

\begin{corollary}
  If $p_i$ increases and all otehr goods' prices are unchanged, then
  the Hicksian demand for good $i$ weakly decreases.
\end{corollary}

\begin{prop}
  Suppose $u$ is a continuous, strictly quasiconcave utility function 
  satisfying local nonsatiation. Then for $i = 1, 2, \dots, L$, we have
  \[
  h_i(p,u) = \frac{\partial}{\partial p_i} e(p,u)
  \]
  As a corollary, if we differentiate once more we get cross partials,
  which must be equal, that is, provided $h$ is continuously
  differentiable:
  \[
  \frac{\partial h_i}{\partial p_j} = \frac{\partial h_j}{\partial p_i}
  \]
\end{prop}

\begin{definition}[Slutsky Matrix]
  The Slutsky substitution matrix $S(p, w)$ is the $L \times L$ matrix with
  \[
  s_{ij} = \frac{\partial x_i}{\partial p_j} + \frac{\partial x_i}{\partial w} x_j
  \]
  which describes the change in the demand $x_j$ resulting from
  changing $p_j$ and giving the consumer extra wealth $x_j
  \cdot \partial p_j$ to make the old bundle affordable.

  As a corollary, the Slutsky substitution matrix is symmetric and
  negative semi-definite. It is derived from $S(p,u) = D_ph(p,u) =
  D^2_pe(p,u)$, which is where the symmetry comes from.
\end{definition}


\begin{prop}[Slutsky Equation]
  Suppose $u$ is a continuous, locally nonsatiated, and strictly
  quasiconcave utility function, and let $w := e(p, u_0)$, then
  \[
  \frac{\partial h_i (p, u_0)}{\partial p_j}
  = \frac{\partial x_i(p,w)}{\partial p_j}
  + \frac{\partial x_i(p,w)}{\partial w} x_j(p,w) 
  = s_{ij}
  \]
  which, after rearranging, gives
  \[
  \frac{\partial x_i(p,w)}{\partial p_j} 
  = \underbrace{\frac{\partial h_i (p, u_0)}{\partial p_j}}_{\text{price effect}}
  - \underbrace{\frac{\partial x_i(p,w)}{\partial w} x_j(p,w)}_{\text{income effect}}
  \]
\end{prop}

\begin{proof}
  From Shephard's Lemma/FOC on the EMP, $h_i(p,u) = \frac{\partial
    e(p,u)}{\partial p_i}$. 


  We know that $h_i(p,u) = x_i(p, e(p,u))$, so we can differentiate
  wrt $p_j$ to get
  \begin{align}
    \frac{\partial h_i(p,u)}{\partial p_j}
    & = \frac{\partial x_i(p, e(p,u))}{\partial p_j}
    + \frac{\partial x_i(p,e(p,u))}{\partial e(p,u)}
    \frac{\partial e(p,u)}{\partial p_j} & \notag \\
    & = \frac{\partial x_i(p, e(p,u))}{\partial p_j}
    + \frac{\partial x_i(p,e(p,u))}{\partial w}
    \frac{\partial e(p,u)}{\partial p_j} & \notag \\
    & = \frac{\partial x_i(p, e(p,u))}{\partial p_j}
    + \frac{\partial x_i(p,e(p,u))}{\partial w}
    h_j(p,u) & \text{by Shephard's Lemma} & \notag \\
    & = \frac{\partial x_i(p, e(p,u))}{\partial p_j}
    + \frac{\partial x_i(p,e(p,u))}{\partial w}
    x_j(p,e(p,u)) & \notag \\
    & = \frac{\partial x_i(p, w)}{\partial p_j}
    + \frac{\partial x_i(p,w)}{\partial w}
    x_j(p,w) & \notag
  \end{align}
\end{proof}
