



\subsection{Preliminaries}
\label{sec:preliminaries}

Consumers are assumed to have utility functions that take quasilinear
form.

\begin{definition}[Quasilinear Utility]
  A utility function is quasilinear with respect to a numeraire
  commodity if it takes the form
  \[
  u_i(x_{i}, h) = g_i(x_{-1i}, h) + x_{1i}
  \]
  where $x_1$ is the numeraire and $x_{-1}$ is all other goods. Thus
  the consumer gets linear utility in the numeraire and will always
  consumer other goods only until on the margin the numeraire is
  preferred, i.e the consumption decision for other goods
  $x_{-1i}(p,h)$ is constant as a function of wealth. We can then
  write the indirect utility function as
  \[
  v_i(p,w,h) = g_i(x_{-1i}(p,h), h) + (w_i - p \cdot x_{-1i}(p, h))
  \]
  there the final term is the amount of money left over for the
  numeraire. If we then let $\phi_i(p,h) := g_i(x_{-1i}(p,h), h) - p
  \cdot x_{-1i}(p, h)$ we get indirect utility of the form
  \[
  v_i(p, w_i, h) = \phi_i(p,h) + w_i
  \]
  which, as above, has no wealth effects.
\end{definition}

We additionally assume that prices of traded goods are unaffected by
any actions within our model, so we assume $p$ is constant and
suppress its notation. That is, we denote $\phi_i(p,h)$ as
$\phi_i(h)$. We further assume that $\phi \in {\cal C}^2$ with
$\phi''_i(\cdot) < 0$. In what follows, we study a two agent model,
where the first consumer makes the choice about how much of the
externality to consume.

The first consumer chooses $h$ to solve
\[
\max_{h \geq 0} \phi_1(h)
\]
which gives, at the optimum,
\[
\phi_1'(h^*) \leq 0 \quad \text{with equality if } h^* > 0
\]
(because she cannot consume negative externality).

By contrast, any Pareto Optimal allocation $h^\circ$ would solve
\[
\max_{h \geq 0} \phi_1(h) + \phi_2(h)
\]
which gives, at the optimum
\[
\phi_1'(h^\circ) \geq -\phi_2'(h^\circ) \quad \text{with equality if } h^\circ > 0
\]

Note that unless $h^* = h^\circ = 0$, as long as $\phi_2(h) \neq 0$
the equilibrium level of $h$ is not Pareto efficient.

\subsection{Solutions to the Externality Problem}
\label{sec:solut-extern-probl}

\subsubsection{Quotas}

Suppose that $h$ generates a negative external effect, so that
$h^\circ < h^*$. Then the government could impose that $h \leq
h^\circ$ and indeed consumer one would set the level of the
externality at $h^\circ$.

\subsubsection{Taxes}

The government could either tax per unit of $h$ or subsidize per unit
reduction of $h$ (on consumer $1$). Suppose consumer $1$ were forced
to pay a tax $t_h$ per unit of $h$. Then she maximizes
\[
\max_{h \geq 0} \phi_1(h) - t_h \cdot h
\]
thus if $t_h = - \phi_2'(h^\circ) > 0$, the competitive equilibrium
will be $\phi_1'(h) = t_h = -\phi_2'(h)$, which is satisfied by
$h^\circ$. This type of tax is known as a Pigouvian corrective tax. In
particular, the optimality-resotoring tax is set at the level of the
marginal externality at the optimal solution.

We could also have used a subsidy above, which has consumer $1$
maximize
\[
\max_h \phi_1(h) + s_h(h^\circ - h)
\]
which is effectively a lump sum transfer combined with a tax, and can
be optimality restoring.

Note that in many cases the government lacks sufficient information
about the costs and benefits of the externality, and also sometimes is
unable to directly tax the externality and instead must apply quotas
or taxes to total output. In these cases, we often cannot achieve an
efficient outcome.

\subsection{Coase}
\label{sec:coase}

Coase viewed the externality problem as arising from a combination of
\begin{enumerate}[(i)]
\item lack of clearly specified property rights
\item transaction costs
\end{enumerate}


\begin{theorem}[Coase]
  If property rights are clearly specified and there are no
  transaction costs, bargaining will lead to an efficient outcome no
  matter how the rights are allocated.
\end{theorem}

Note that this does not say that distribution is unaffected, just that
an efficient outcome is attained.

As an example, consider allocating to consumer $2$ rights to
externality free environment. She can make a take it or leave it offer
of payment $T$ (that consumer $1$ would make to her) in order for $1$
to have the rights to consume the externality at level $h$ (but
consumer $1$ may reject the deal). Thus consumer $2$ maximizes
\[
\max_{h, T} \phi_2(h) + T \quad s.t. \quad \phi_1(h) - T \geq \phi_1(0)
\]
but since the constraint is binding, we can solve for $T$ and plug in
to get the reframed problem for consumer $1$:
\[
\max_h \phi_2(h) + \phi_1(h) - \phi_1(0)
\]
which yields the FOC that guarantees the socially optimal level.

If we switched the property rights, such that consumer $1$ has the
right to generate as much $h$ as she wants. Still consumer $2$ makes a
take it or leave it offer to consumer $1$ offering payment $T$ in
order to reduce the externality to $h$. Then consumer $1$'s problem is
\[
\max_{h,T} \phi_2(h) - T \quad s.t. \quad \phi_1(h) + T \geq \phi_1(h^*)
\]
again the constraint is binding so we can rewrite the above to
\[
\max_h \phi_2(h) - (\phi_1(h^*) - \phi_1(h))
\]
which again gives FOCs that generate the social optimum.

Note that in special case where two parties are firms, another
solution to let the two firms merge. The resulting firm would fully
internalize the externality when maximizing profits. Note also that
unlike the tax and quota schemes, Coase’s approach only requires that
the consumers know each others preferences, and not government.



\subsection{Myerson Satterthwaite}
\label{sec:myers-satt}

Suppose $h$ is now a discrete choice, $\{0, \bar h\}$ and $\theta,
\eta$ random variables that influence each agent's utility, i.e.
$\phi_1(h; \theta)$ and $\phi_2(h; \eta)$. Then we can define
\[
b(\phi) = \phi_1(\bar h; \theta) - \phi_1(0; \theta) > 0
\]
and
\[
s(\phi) = \phi_2(0; \eta) - \phi_2(\bar h; \eta) > 0
\]
thus when $b(\theta) > s(\eta)$ the Pareto optimal level is $\bar
h$. Then if we let $G(b)$ and $F(s)$ be the CDFs induced by $\theta,
\eta$, and we give consumer $2$ the right to an externality free
environment, then she will always choose $h=0$ which is inefficient if
$b(\theta) > s(\eta)$.

Suppose now we allow consumer $1$ to offer a transfer $T$ for the
ability to consume $\bar h$. And assume each consumer knows their own
values but not the other person's. Under this transfer, we know
consumer $2$ will agree iff $T \geq s$, the probability of which is
$F(T)$. Hence consumer $1$ solves
\[
\max_T \underbrace{F(T)}_{P(\text{2 accepts})}
\underbrace{(b - T)}_{\text{payoff}}
\]
and will make an offer $T$ that is less than $b$ (positive profits)
but is greater than the minimum value of $s$ (otherwise agent $2$
never accepts). But suppose given the realization of $\eta$ that 
\[
b(\theta) > s(\eta) > T
\]
then in this case the efficient level is $\bar h$ since $b(\theta) >
s(\eta)$, but the offer will be rejected. Hence there is a positive
probability of inefficiency in this context even with transfers.

\begin{prop}[Myerson-Sattherthwaite]
  No bargaining procedure can lead to an efficient outcome for all
  values of $b$ and $s$ when they are private information and
  independently distributed.
\end{prop}

\subsection{Missing Markets}
\label{sec:missing-markets}

Suppose we can construct a competitive market for the externality
(which may not always be possible), with price per unit $p_h$. Then
consumer $1$ buys rights to $h$ to solve
\[
\max_{h_1} \phi_1(h_1) - p_h h_1
\]
and consumer $2$ sells rights to $h$ to solve
\[
\max_{h_2} \phi_2(h_2) + p_h h_2
\]
which gives the two FOCS
\[
\phi_1'(h_1) = p_h \quad \text{and} \quad \phi_2'(h_2) = -p_h
\]
which clearly generateds the Pareto Optimal level of the externality
in equilibrium (when we impose $h_1 = h_2 = h^{**}$. 


\subsection{Prices versus Quantities}
\label{sec:pric-vers-quant}



We saw earlier that taxes and quotas generated efficient levels of the
externality. Yet when we add uncertainty, the two are not equivalent
any longer.

\subsubsection{Setup}


Suppose that firms generate externalities (and can be regulated) and
consumers derive negative benefit from the externalities. Suppose
there is uncertainty in the value of the externality, with the firm's
profit given by $\pi(h, \theta)$ and the consumer's utility $\phi(h,
\eta)$, where $\theta, \eta$ are random variables that are privately
observed. However, the CDFs of $\theta, \eta$ are publicly known
\textit{ex ante}. We assume that $\pi(h,\theta)$ and $\phi(h, \eta)$
are strictly concave in $h, \forall \theta, \eta$.

If the government could obesrve realizations of $\theta$ and $\eta$
and make its decision \textit{ex post} then it would get the first
best quota
\[
h^\circ(\theta, \eta) = \arg \max_h \{ \pi(h, \theta) + \phi(h, \eta) \}
\]
or tax
\[
t_h^\circ(\theta, \eta) 
= \frac{\partial \pi(h^\circ(\theta, \eta), \theta)}{\partial h}
= \frac{\partial \phi(h^\circ(\theta, \eta), \theta)}{\partial h}
\]
these are the first best solutions, and form the basis of
comparison. Both will yield equilibrium externality levels of
$h^\circ$, given above.


However, the government is often unable to observe these values, or at
least unable to craft policy in response to every realization. So now
we assume the government must commit in advance to a tax or quota
policy, knowing only the \textit{ex ante} distributions of $\theta$
and $\eta$. 

We must first consider the firm's response to taxes or quantity
regulation. In the case of quantity regulation with quata $\hat h$,
the firm solves
\[
\max_{h \geq 0} \pi(h, \theta) \quad s.t. \quad h \leq \hat h
\]
and we denote the optimal choice $h^q(\hat h, \theta)$. In the case of
a tax, the firm solves
\[
\max_{h \geq 0} \phi(h, \theta) - th
\]
and we denote the optimal choice $h^t(t, \theta)$. 


We now consider the optimal tax or quota from the government's
perspective. Under quota regulation the planner solves
\[
\hat h^* = \arg \max_{\hat h}
E \left[ \pi(h^q(\hat h, \theta), \theta) 
+ \phi(h^q(\hat h, \theta), \eta) \right]
\]
and in the case of taxation the planner solves
\[
t^* = \arg \max_t
E \left[ \pi(h^t(t, \theta), \theta)
+ \phi(h^t(t, \theta), \eta) \right]
\]

Note that these optimizations are performed in expectation, so in
general it will not be the case that for any given realization they
are first best. Thus we will want to study the loss in aggregate
surplus under given realizations of $(\theta, \eta)$. In the case of a
quota, the loss is
\[
(\pi(h^q(\hat h, \theta), \theta) + \phi(h^q(\hat h, \theta), \eta))
- (\pi(h^\circ(\theta, \eta), \theta) + \phi(h^\circ(\theta, \eta), \eta))
\]
which is equivalent to
\[
\int_{h^\circ(\theta, \eta)}^{h^q(\hat h, \theta)}
\left( \frac{\partial \pi(h, \theta)}{\partial h}
+ \frac{\partial \phi(h, \eta)}{\partial h} \right) dh
\]
and in the case of a tax the loss is
\[
(\pi(h^t(t, \theta), \theta) + \phi(h^t(t, \theta), \eta))
- (\pi(h^\circ(\theta, \eta), \theta) + \phi(h^\circ(\theta, \eta), \eta))
\]
which is equivalent to
\[
\int_{h^\circ(\theta, \eta)}^{h^t(t, \theta)}
\left( \frac{\partial \pi(h, \theta)}{\partial h}
+ \frac{\partial \phi(h, \eta)}{\partial h} \right) dh
\]


Weitzman makes functional form simplifications which are similar to
second order Taylor approximations around the point $\hat h^*$ to give
\[
\pi(h, \theta) 
= b(\theta) + (\pi' + \beta(\theta))(h - \hat h^*) + \frac{\pi''}{2}(h - \hat h^*)^2
\]
and
\[
\phi(h, \theta) 
= a(\eta) + (\phi' + \alpha(\eta))(h - \hat h^*) + \frac{\phi''}{2}(h - \hat h^*)^2
\]
with
\[
E[\beta(\theta)] = E[\alpha(\eta)] = 0
\]
and he defines the advantage of a tax over quota regulation as
\[
\Delta 
= E[\{ \pi(h^t(t, \theta), \theta) + \phi(h^t(t, \theta), \eta) \}
- \{ \pi(\hat h, \theta) + \phi(\hat h, \eta) \} ]
\]
to give the result

\begin{prop}[Weitzman]
  The benefit of price regulation over quantity regulation, as given
  above, is
  \[
  \Delta = \frac{\sigma^2}{2(\phi'')^2}(\pi'' - \phi'') = (+)[(-) + (+)]
  \]
  where $\sigma^2$ is the variance of $\phi$ at $\hat h$. This means
  that price regulation is better than quantity regulation if and only
  if $\phi'' > |\pi''|$. The intuition is that prices are better when
  the firm faces more uncertainty, since they allow variability here,
  while quantities are better when the consumer faces more
  uncertainty, since it imposes a hard cap on the externality.
\end{prop}

\subsection{Equilibrium Number of Boats}
\label{sec:equil-numb-boats}

This is an example that is addressed both in lecture and in the
problem set, so see either for full exposition and solution. Basically
the planner cares about average cost of sending out additional boats
while the individual fisherman only care about whether they can make
positive profit on the margin (they don't internalize the externality
that they impose on the other fishermen by reducing the number of fish
per capita or something). Thus the planner's number of boats is always
smaller than if we let the fishermen decide on their own.



** TODO: do we need to know Lindahl equilibrium? I think it was on a problem set, and is addressed briefly on MWG pp. 363--364 **

** TODO: write up definitions of depletable and non-depletable externalities MWG section 11.D **




