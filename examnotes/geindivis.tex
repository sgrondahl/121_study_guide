




\subsection{Housing Market: Basic Exchange Economy}
\label{sec:hous-mark-basic}

\begin{definition}[Housing Market]
  A housing market is a list $((a_k, h_k)_{k \in 1, \dots, n}, P)$
  with $a_k$ agent $k$ who intially owns house $h_k$. Each agent has
  strict partial preference ordering $P_a$ over houses. Let $R_a$ be
  the weak preference relation associated with these strict
  preferences.
\end{definition}

\begin{definition}[Allocation]
  An allocation in a housing market is a matching $\mu : A \to H$, a
  bijective function with house $\mu(a)$ the house assigned to agent
  $a$.
\end{definition}

\begin{definition}[Pareto Efficient]
  A matching $\mu$ is Pareto efficient if there is no other matching
  $\nu$ such that $\nu(a) R_a \mu(a) \; \forall a \in A$, and $\nu(a)
  P_a \mu(a)$ for some agent $a \in A$.
\end{definition}

\begin{definition}[Core]
  The core is the set of matchings $\mu$ such that $\nexists $
  coalition $B \subset A$ and matching $\nu$ such that 
  \begin{enumerate}[(a)]
  \item $\forall a \in B, \nu(a) = a_l$ for some $a_l \in B$, and
  \item $\forall a \in B, \nu(a) R_a \mu(a)$ and $\exists a \in B :
    \nu(b) P_a \mu(a)$
  \end{enumerate}
  (i.e. it can't be the case that the coalition can exchange homes
  without its initial endowment to improve itself).
\end{definition}

    
A housing market can be thought of as a non-transferable utility (NTU)
cooperative game. For these types of games, the core is the central
solution concept for NTU-cooperative games.


\begin{definition}[Individually Rational]
  A matching is individually rational if every agent receives a house
  at least as good as her initial house.
\end{definition}

\begin{prop}
  A core matching is individually rational \textit{and} Pareto efficient. 
\end{prop}

\begin{proof}
  For the former, consider the set of coalitions of singletons. For
  the latter, consider the grand coalition.
\end{proof}

\begin{theorem}[Shapley-Scarf 1974]
  The core is nonempty for each housing market and it is unique.
\end{theorem}

\begin{proof}
  Constructive: use Gale's Top Trading Cycles (TTC) algorithm. ** TODO **
\end{proof}


** TODO: just see lecture notes here **