





The monopoly pricing problem is
\[
\max_q \pi(q)
\]
where $\pi(q) = q \cdot P(q) - c(q)$, and we assume it to be
differentiable, concave, and having an interior solution. Thus the FOC
is
\[
q^m \cdot P'(q^m) + P(q^m) = c'(q^m)
\]

\begin{definition}[Lerner Index]
  The Lerner Index is given by
  \[
  \frac{P(q^m) - c'(q^m)}{P(q^m)}
  \]
\end{definition}

\begin{prop}[Inverse Elasticity Rule]
  The percentave markup in the monopoly model is
  \[
  \frac{P(q^m) - c'(q^m)}{P(q^m)} = - \frac{1}{\epsilon}
  \]
  where $\epsilon = \frac{dQ}{dP} \cdot \frac{P}{Q}$ is the price
  elasticity of demand.
\end{prop}


\subsection{Assumptions}
\label{sec:assumptions}

Assume that
\begin{itemize}
\item[(A1)] Indirect utility has the form
  \begin{align}
    V_i(p,w) & = \max_{q : p \cdot q \leq w} u_i(q_1, \dots, q_L) & \notag \\
    & = \max_{q_1} f_i(q_1) + [w - p_1 q_1] & \notag
  \end{align}
  which gives us that Marshallian demand has no income effect, so we
  can write $x(p)$ instead of $x(w,p)$ and we can make comparisons
  across consumers with different levels of wealth
\item[(A2)] $w_i = w_{0i} + s_i \pi(p)$ which means that firms are
  owned by consumers, where $s_i$ is consumer $i$'s share of the
  profits that the firm obtains
\end{itemize}

\subsection{Monopolist Quantity}
\label{sec:monopolist-quantity}



\begin{prop}
  Given any depand curve $P(q)$ and cost function $c(q, \theta)$, define
  \[
  q^m = \sum \arg \max_q q \cdot p(q) - c(q, \theta)
  \]
  If $c(q, \theta)$ has increasing differences, then $q^m$ is weakly
  increasing in $\theta$.
\end{prop}

\begin{prop}[Monopolistic Underprovision]
  Let
  \[
  q^{FB} = \sup \arg \max_q \int_0^q P(s) ds - c(q)
  \]
  and
  \[
  q^m = \sup \arg \max_q q \cdot P(q) - c(q)
  \]
  If $P'(q) \leq 0$ then $q^m \leq q^{FB}$.
\end{prop}



\subsection{Monopolist Quality}
\label{sec:monopolist-quality}

Suppose a monopolist chooses quality $s$ in addition to quantity, so
now we can define
\[
W(q,s) = \int_0^q P(x,s) dx - c(q,s)
\]
and
\[
\pi(q,s) = q \cdot P(q,s) - c(q,s)
\]

\begin{prop}
  Let 
  \[
  s^{FB} = \sup \arg \max_s W(q,s)
  \]
  and
  \[
  s^m(q) = \sup \arg \max_s \pi(q,s)
  \]
  If $P(q,s)$ has decreasing differences in $q$, then $s^{FB}(q) \geq
  s^m(q)$.
\end{prop}

The intuition here is that the value of quality is less for the
marginal consumer than for the average consumer. Since the monopolist
only cares about the marginal consumer but the social planner cares
about the average consumer, we get a distortion from optimum, known as
the \textbf{Spence Distortion}. On the other hand, the IO literature
notes that since the monopolist sells fewer units, the marginal
monopolist consumer has a higher value than the social planner's
marginal consumer. Thus in richer models where $q$ can vary (in the
above we've fixed $q$), this effect can offset and yield $s^m >
s^{FB}$.




