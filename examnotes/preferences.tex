A utility function serves two purposes:

\begin{enumerate}
\item Positive -- used to predict behavior (utility maximization)
\item Normative -- used to determine what is `good' (welfare analysis)
\end{enumerate}

\begin{definition}[Choice Space]
  Let a choice space $X \subset \R^N$ be the set of states of
  the world (in an abstract sense).
\end{definition}


\begin{definition}[Utility Function]
  A utility function is a map $u: X \to \R$.
\end{definition}

\begin{definition}[Preference]
  A preference is a set $P \subset X \times X$ with $x \succeq_p y$ if
  $(x,y) \in P$.
\end{definition}

\begin{definition}
  The utlitity function $u: X \to \R$ \textbf{represents} a preference
  $P$ if $u(x) \geq u(y) \iff x \succeq_p y$.
\end{definition}

Note: In general, representations are not unique. In fact, given any
strictly increasing function $f: \R \to \R$ and some $u$ representing
$P$, $f \circ u$ also represents $P$.

\begin{definition}[Complete]
  A preference relation $\succeq_p$ is complete on $X$ if $\forall x,y
  \in X$ either $x \succeq_p y$ or $y \succeq_p x$.
\end{definition}

\begin{definition}[Reflexive]
  A preference relation $\succeq_p$ is reflexive on $X$ if $\forall x
  \in X$, $x \succeq_p x$.
\end{definition}

\begin{definition}[Transitive]
  A preference relation $\succeq_p$ is transitive on $X$ if $\forall
  x,y,z \in X$, $(x \succeq_p y) \wedge (y \succeq_p z) \implies x
  \succeq_p z$.
\end{definition}

\begin{prop}
  If $\succeq_p$ is represented by $u$ on $X$, then $\succeq_p$ is
  complete, reflexive, and transitive.
\end{prop}

\begin{proof}
  Prove each part separately:
  \begin{enumerate}
  \item Complete: $\forall x,y \in X, u(x) \geq u(y)$ or $u(y) \geq
    u(x)$ (property of $R$), $\implies x \succeq_p y$ or $y
    \succeq_p x$.
  \item Reflexive: $\forall x \in X, u(x) \geq u(x)$ (function well
    defined) $\implies x \succeq_p x$.
  \item Transitive: $x \succeq_p y, y \succeq_p z \implies u(x)
    \geq u(y)$ and $u(y) \geq u(z)$ (because $u$ is
    representative). Then by transitivity on $\R$ we have $u(x) \geq
    u(z)$, and again because $u$ is representative, $x \succeq_p z$.
  \end{enumerate}
\end{proof}


Can preferences reasonably violate these properties? Consider each one
individually:
\begin{itemize}
\item Intransitive preferences permit a cycle $x \to y \to z \to x$
  such that an agent is being made better off with each exchange, so
  this would permit a small charge to be extracted at each exchange
  without ultimately changing the agent's choice, which is referred to
  as the ``Dutch Book.'' It is thought that since we don't observe
  this in practice that such preferences must not exist, but it is
  possible that some other frictions exist.
\item Anything without reflexivity makes not sense at all, so we don't
  consider it.
\item Some preferences might not be complete, generally because of:
  \begin{enumerate}
  \item moral issues -- agents might refuse to establish a preference
    over certain alternatives for moral issues (e.g. killing one vs
    killing 100)
  \item bounded rationality -- might not even be able to fully
    consider all possible states of the world, so can't have complete
    preferences
  \end{enumerate}
\end{itemize}

As an aside, consider a famous experiment that seems to isolate
preferences which violate transitivity. Kahneman and Tversky (1984) do
(basically) the following: Tell people they are going to MIT Coop to
buy \$125 stereo and \$15 calculator. Then salesman either says that
the calculator or stereo is on sale at the Harvard Coop for \$5
cheaper. Most people choose to travel to Harvard Coop to get \$5 if it
is taken off calculator, but most don't if it is taken off stereo
(they just buy the stuff at the MIT Coop). A third treatment has the
salesman say that they are out of stock of both items, but if people
go to the Harvard Coop (where the stuff is in stock), they can choose
to take \$5 off of either item, and people are indifferent about which
item to discount. Thus if we consider $x$ to be getting \$5 off
calculator by buying at Harvard Coop, $y$ to be getting \$5 off stereo
by buying at Harvard Coop, and $z$ just buying both at MIT, the first
two treatments give us that $x \succeq z$ and $z \succeq y$, but the
third treatment gives us that $x ~ y$, a contradiction (since
transitivity) should have given us taht $x \succeq y$. The finiding is
controversial, and it is often argued that people only make this
mistake because they are uneducated about/unfamiliar with this
situation. In fact, John List (Chicago) has a bunch of papers dealing
with baseball card markets, and he finds that experience limits
irrational/paradoxical behaviors. See also Gul and Pesendorfer, ``The
Case for Mindless Economics,'' which is a response to neuroeconomics.


Suppose we have the three properties we've been studying, then is the
converse of the previous proposition true, i.e. can we always find a
$u$ that represents $\succeq_p$?

\begin{prop}
  Let the choice space $X \subset \R^N$ be finite, $P$ a preference on
  $X$ with $\succeq_p$ complete, reflexive, and transitive on $X$, then 
  there exists a utlity function $u$ that represents $P$.
\end{prop}

\begin{proof}
  Let $B(x) := \{ z\ in X : x \succeq_p z \}$, with $B$ meaning
  ``below.'' Then we can define $u(x) = |B(x)|$, which is well defined
  since $X$ is finite (note that the worst item $w \in X$ has $u(w) =
  |B(w)| = 1$ by reflexivity. Prove each direction of representation:
  \begin{enumerate}
  \item Show that $u(x) \geq u(y) \implies x \succeq_p y$. Note
    first that $u(x) \geq u(y) \implies |B(x)| \geq |B(y)|$. There
    are two possibilities:
    \begin{enumerate}
    \item $y \in B(x)$, meaning (by construction of $B$) that $x
      \succeq_p y$, or
    \item $y \not\in B(x) \implies |B(x)| = |B(x) \setminus \{y\}|$,
      and using reflexivity we have $y \in B(y) \implies |B(y)| - 1 =
      |B(y) \setminus \{y\}|$. Moreover, $|B(x)| \geq |B(y)| \implies
      |B(x) \setminus \{y\}| = |B(x)| > |B(y)| - 1 = |B(y) \setminus
      \{y\}|$. Hence there must be some $z \in X \setminus \{y\}$ such
      that $z \in B(x)$ but $z \not\in B(y)$, i.e. $x \succeq_p z$ but
      $y \not\succeq_p z$. Using completeness of $P$, though, we
      therefore have $z \succeq_p y$, therefore by transitivity we
      have $x \succeq_p y$, meaning $y \in B(x)$ (by construction), a
      contradiction.
    \end{enumerate}
  \item Show that $x \succeq_p y \implies u(x) \geq u(y)$. Suppose
    that in fact $x \succeq_p y$. Then $\forall z \in X, z \in B(y)
    \implies y \succeq_p z \implies x \succeq_p z$ (by transitivity)
    $\implies z \in B(x)$ (by construction), which in combination
    gives $z \in B(y) \implies z \in B(x)$, hence $B(y) \subseteq
    B(x)$ and therefore $|B(y)| \leq |B(x)|$, i.e. $u(y) \leq u(x)$.
  \end{enumerate}
\end{proof}

\begin{definition}[Monotone Preference Relation]
  A preference relation $\succeq$ on $X$ is monotone if $\forall x,y \in X, y
  \geq x \implies y \succeq x$.
\end{definition}

\begin{definition}[Strictly Monotone Preference Relation]
  A preference relation $\succeq$ on $X$ is strictly monotone if $\forall x,y
  \in X$ with $x \neq y$, $y \geq x \implies y \succ x$.
\end{definition}

\begin{definition}[Locally Non-Satiated Preference Relation]
  The preference relation $\succeq$ on $X$ is locally non-satiated if
  $\forall x \in X$, and $\forall \delta > 0, \exists y \in X$
  s.t. $||y-x|| < \delta$ and $y \succ x$.
\end{definition}

\begin{definition}[Convex Preference Relation]
  The preference relation $\succeq$ on $X$ is convex if $\forall x,y
  \in X$, $(y \succeq x) \wedge (z \succeq x) \wedge (y \neq z) \implies
  \forall \alpha \in (0,1), \alpha y + (1-\alpha)z \succeq x$.
\end{definition}

\begin{definition}[Continuous Preference Relation]
  The preference relation $\succeq$ on $X$ is continuous if it is
  preserved under limits: for any sequence $\{(x^n,y^n)\}_{n=1}^\infty
  $ with $x^n \succeq y^n \forall n$, $x \succeq y$ where $x, y$ are
  limit points of the respective sequences.
\end{definition}

Note: lexicographic preferences are not continuous. Consider the
elements of the choice space with (ordered) components, $x = (x_1,
x_2), y = (y_1, y_2)$. Then $\succeq$ is defined by $x \succeq y$ if
$x_1 > y_1$ or $x_1 = y_1$ and $x_2 > y_2$. Then if we consider $x^n =
(\frac{1}{n}, 0), y^n = (0, 1)$, we have $x^n \succeq y^n$ but $y (= (0,1))
\succeq x (= (0,0))$.

\begin{theorem}[Debreu's Theorem]

  Suppose the rational preference relation $\succeq$ on $X$ is
  continuous. Then there is a continuous utility function $u(x)$ that
  represents $\succeq$.
  
\end{theorem}

** TODO: do we need proof of Debreu?? **

\begin{definition}[Directly Revealed]
  At given price vector $p$, if consumption bundle $x$ is chosen when
  $y$ could have been chosen, we say $x$ is directly revealed
  preferred to $y$, $x R^D y$.
\end{definition}

\begin{definition}[Indirectly Revealed]
  If a sequence of direct comparisons indicates that $x$ is preferred
  to $y$, then $x$ is revealed preferred to $y$. That is, if
  \[
  xR^Dz_1, z_1R^Dz_2, \dots, z_nR^Dy
  \]
  when we write $xRy$.
\end{definition}

\begin{definition}[GARP]
  A set of consumption choices $\{(p^1,x^1), \dots, (p^n, x^n)\}$
  satisfies the general axiom of revealed preferences (GARP) if and
  only if
  \[
  x^i R x^j \implies \neg (x^j P^D x^i)
  \]
\end{definition}

\begin{prop}[Arfiat 1967]
  Let $\{(p^1,x^1), \dots, (p^n,x^n)\}$ be a finite set of consumption
  choices.  If a finite set of demand data violates GARP, then tehse
  data are inconsistent with choice according to locally nonsatiated,
  complete, and transitive preferences.

  Conversely, if a finite set of demand data satisfies GARP, then
  these data are consistent with choice according to strictly
  increasing, continuous, convex, complete, and transitive
  preferences.
\end{prop}

Arfiat is huge because it gives succinct, testable conditions that a
finite dataset must satisfy to be consistent with utility
maximization.