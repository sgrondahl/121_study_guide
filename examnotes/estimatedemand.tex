

Our goal is to estimate \textit{indidivual} demand $x =
f(p,w)$. Rationality imposes some restrictions (Slutsky symmetry, HOD
0, Walras' Law), but still goods space is very high dimensional. In
general, we can think of goods in two ways:

\begin{itemize}
\item[(product space)] consider specific types of products
\item[(characteristic space)] consider characteristics of goods to
  reduce dimensionality
\end{itemize}

\subsection{Goods in Product Space}
\label{sec:goods-product-space}

\subsubsection{Product Grouping}
\label{sec:product-grouping}

We can divide goods into groups, and then assume the consumer employs
multi-stage budgeting: first she allocates expenditures across groups,
then she allocates expenditure within groups. If demand is weakly
separable, i.e. $u(x) = U(u_1(g_1), \dots u_n(g_n))$, then demand
within a group is independent of prices outside the group (Deaton and
Muelbauer (1980)).

We can also approach grouping goods as imposing price restrictions, as
we did on the second problem set. In this case, we consider two groups
of goods $x$ and $y$ with price vectors $p$ and $q$, respectively. If
we assume that prices $q$ move in unison, i.e. $q = \alpha q_0$, then
we can consider the group $y$ as a \textit{single composite good} $z =
q_0 \cdot y$ and redefine the utility function to get
\[
\tilde u(x, z) = \max_y u(x,y) \quad s.t. \quad q_0 \cdot y \leq z
\]
The interpretation for the above is that $z$ is the amount spent on
group $y$, and for each level of expenditure $z$ we maximize utility
over within-group bundle $y$. Hence we have a two stage selection --
first group-level expenditure then within group expenditure. In the
problem set, we proved that under the assumptions this approach is
dual to directly maximizing over $x$ and $y$ subject to the budget
constraint, i.e. an allocation $(x,y)$ is a solution to the
compositive goods problem if and only if it is also a solution to the
direct maximization problem 
\[
\max_{x,y} f(x,y) \quad s.t. \quad p \cdot x + q \cdot y \leq w
\]

\subsubsection{Restricting Preferences}
\label{sec:restr-pref}

We can also improve tractibility of demand estimation by imposing
functional form assumptions on preferences. In particular, we consider
the following:

\begin{definition}[Quasi-Homothetic Demand]
  Demand is quasi-homothetic if it takes the following form:
  \[
  x(p,w) = \alpha(p) + w \cdot \beta(p)
  \]
\end{definition}

\begin{prop}[Gorman 1961]
  Demand is quasi-homothetic if and only if the expenditure function
  takes the form:
  \[
  e(p, u) = a(p) + u\cdot b(p)
  \]
  where $a(p)$ and $b(p)$ are homogenous of degree $1$.
\end{prop}

Note: Gorman form is particularly convenient because we have
established an explicit relation between $\alpha, \beta$ and $a,b$
(see section \ref{sec:gorman-polar-form}, hence once we estimate
demand we are automatically handed the Hicksian without having to
solve an ugly system of ODEs, thus allowing us to easily do welfare
calculations.


A special case of quasi-homothetic preferences with gives explicit
functional form is Stone-Geary Preferences:

\begin{definition}[Stone-Geary Preferences]
  Stone-Geary preferences are represented by a utility function of the form
  \[
  u(x_1, \dots, x_L) = \prod_{i=1}^L(x_i - \gamma_1)^{\beta_i}
  \]
  with
  \[
  \sum_i \beta_i = 1
  \]
\end{definition}

Note that Cobb-Douglas utility is a special case of the above with
$\gamma_i = 0$.

\begin{prop}
  Stone-Geary preferences generate Marshallian demand
  \[
  x(p,w) \; : \; p_ix_i = p_i\gamma_i + \beta_i(w - \sum_j p_j\gamma_j)
  \]
  and Hicksian demand
  \[
  h(p,u) \; : \; p_ih(p, u_0) = p_i\gamma_i + \beta_i\beta_0 \prod_jp_j^{\beta_j}u_0
  \]
  where $\beta_0$ is a function of the parameters. Note that the
  amount spent is \textbf{linear} in both price and wealth, hence this
  is also known as the linear demand system. The expenditure function,
  which can be used for welfare analysis, is
  \[
  e(p,u) = \sum_j p_j\gamma_j + \beta_0 \prod_j p_j^{\beta_j}u_0
  \]
\end{prop}

\subsubsection{Almost Ideal Demand System (AIDS)}
\label{sec:almost-ideal-demand}

Deaton and Muelbauer (1980) model expenditure functions as
quasi-homothetic:
\[
\log e(p,u) = a(p) + ub(p)
\]
which implies shares of the form
\[
s_i 
= \frac{p_ix_i}{w}
= A_i(p) + B_i(p) \log w
\]

Without getting into specifics, the AIDS model is good because it is
very flexible: in general, the AIDS system is a \textbf{first-order
  approximation} to any demand system. Moreover, since it is a system
of expenditure functions, it is easy to do welfare analysis, and is
realtively easy to aggregate over consumers. The downside (``almost
ideal'') is that there are way too many parameters that show up once
we write down Deaton and Muelbauer's proposed functional forms. The
parameters $\gamma_{ij}$ are cross price elasticities, thus if we can
impose restrictions on these we can get a more tracible result.

\subsection{Goods in Characteristic Space}
\label{sec:goods-char-space}

Good are seen as bundles of characteristics and consumers derive
utility from these characteristics and some idiosyncratic taste.  That
is, there are many products, with each product being a unique
combination of different characteristics, and each sumer will choose
the single option that maximizes utility. These models are often
called discrete choice models (choosing either to buy or not to buy),
or ``random utility'' models, because utility is described as a random
variable reflecting unobserved taste differences. Luce first
formalized this class of models with three axioms: (1) choice
probability is strictly greater than zero for all choices, (2) IIA,
(3) separability.

To gain tractibility, we consider \textbf{logit} models. The standard
multinomial logit model assumes $\epsilon_{ij}$ is iid according to
the Extreme Value Type I distribution, $F(\epsilon) =
\exp\{-e^{-\epsilon}\}$. This is due to McFadden (1973) and allows us
to use the Luce (logit) model with random utility, and thus we can
estimate Luce values as a function of observed characteristics. If we
suppose the direct utility individual $i$ gets from product $j$ is
$U_{ij} = V_{ij} + \epsilon_{ij}$, and we let $d_{ij} = 1$ if the
individual $i$ chooses product $j$ and $0$ otherwise, we get the nice
functional form
\[
P(d_{ij} = 1) 
= \frac{\exp(V_{ij})}{\sum_{k} \exp(V_{ik})}
\]


The issue is that logit relies very heavily on the Independence of
Irrelevant Alternatives (IIA) property, i.e. that the odds of choosing
$j$ over $k$ is independent of which other alternatives are available,
i.e.
\[
\frac{P(d_{ij} = 1)}{P(d_{ik}=1)} = \frac{e^{V_{ij}}}{e^{V_{ik}}} \; \forall j,k
\]

However, this is a very strong restriction, and is often
violated. Suppose there are three options to get to school: (1) drive,
(2) take the red bus, (3) take the blue bus. Assume further that the
red/blue bus are otherwise identical (except for color) so they
generate the same utility for me. Suppose then the choice
probabilities are $(0.50, 0.25, 0.25)$ (I am indifferent between the
busses, so I just split my $P(bus) = 0.5$ evenly among them). IIA
would require
\[
\left. \frac{P(car)}{P(red bus)} \right|_{3 choices}
\left. \frac{P(car)}{P(red bus)} \right|_{2 choices}
= \frac{2}{1}
\]
thus when I remove the blue bus, it would require the choice
probabilities be $(0.66, 0.33)$. But in reality because I didn't
really care, we would intuitively expect them to be $(0.5, 0.5)$,
i.e. now that the blue bus is gone I still want to ride the bus $50\%$
of the time, but now I'll just ride the red bus whenever I decide to
bus it (since I don't care about the color).

We can get away from IIA by trying different functional forms (GEV
functional form), nested logits (partition goods into nests, then
choices are MNL conditional on being in a nest, while choices across
nests have logit form), or mixed logits (allow consumer preferences to
deviate from broad population preferences -- sometimes called random
coefficients logit).
