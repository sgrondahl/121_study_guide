

\subsection{Welfare Comparisons}
\label{sec:welfare-comparisons}

Note: in what follows, the superscripts indicate time periods, while
the subscripts denote individual goods. Hence $p^x_y$ would be the
price for good $y$ in period $x$.

\begin{definition}[Equavalent Variation]
  Let $u^0 = v(p^0, w)$ and $u^1 = v(p^1, w)$. Then the equivalent
  variation EV is
  \[
  EV(p^0, p^1, w) = e(p^0, u^1) - e(p^0, u^0) = e(p^0, u^1) - w
  \]
  (note that $e(p^0, u^0) = e(p^1, u^1) = w$). 

  This can be thought of as the change in the consumer's wealth that
  would be equivalent to the price change. That is, she would be
  indifferent between a price change from $p^0 \to p^1$ and a wealth
  change $EV(p^0, p^1, w)$, which of course can be negative.
\end{definition}

\begin{definition}[Compensating Variation]
  Let $u^0 = v(p^0, w)$ and $u^1 = v(p^1, w)$. Then the compensating
  variation CV is
  \[
  CV(p^0, p^1, w) = e(p^1, u^1) - e(p^1, u^0) = w - e(p^1, u^0)
  \]
  (note that $e(p^0, u^0) = e(p^1, u^1) = w$). 

  This can be thought of as the net revenue of the planner who must
  compensate the consumer for the price change after it occurs, to
  keep her utility exactly $u^0$ (again, this could be negative).
\end{definition}

\begin{theorem}[Hicksian-Expenditure Theorem]
  Suppose only the price of good $1$ changes, ceteris paribus, then
  \[
  EV(p^0, u^0, w) 
  = -\int_{p_1^0}^{p_1^1} h_1(p_1, \bar p_{-1}, u^1) dp_1
  \]
  and
  \[
  CV(p^0, u^0, w) 
  = -\int_{p_1^0}^{p_1^1} h_1(p_1, \bar p_{-1}, u^0) dp_1
  \]
  where $\bar p_{-1}$ indicates the prices of all other goods, which
  are not changing.
  
\end{theorem}

To estimate the benefit of a new good we just take the above integrals
out to $\infty$ (or some high price where demand would be zero).

The difficulties with the above is that they rely upon Hicksian
demand, which we can't observe. So we have three tricks we can use to
make welfare analysis tractable:

\begin{enumerate}
\item Assume no income effects in demand, i.e. $x(p, w) = x(p)$ (which
  may be true for small price changes). Then
  \[
  EV = CV = -\int_{p_1^0}^{p_1^1} x_1(p_1, \bar p_{-1}) dp_1
  \]
\item Go from Marshallian to Hicksian demand nonparametrically
  (Slutsky lets us to Marshallian to derivatives of Hicksian, so if we
  solve ODE system we can recover Hicksian)
\item Convert from Marshallian to Hicksian by imposing tractable
  functional form
\end{enumerate}


\subsection{Price Indexes}
\label{sec:price-indexes}

Setup: At $t=0$ we see price $p^0$ and consumption $x^0$.  At $t=1$ we
see price $p^1$ and consumption $x^1$.

\begin{definition}[Laspeyres Price Index]
  The Laspeyres Price Index is 
  \[
  LPI = \frac{p^1 \cdot x^0}{p^0 \cdot x^0}
  \]
\end{definition}

\begin{definition}[Paasche Price Index]
  The Paasche Price Index is 
  \[
  PPI = \frac{p^1 \cdot x^1}{p^0 \cdot x^1}
  \]
\end{definition}

\begin{prop}
  The LPI overstates inflation:
  \[
  \frac{p^1 \cdot x^0}{p^0 \cdot x^0}
  \geq \frac{e(p^1, u^0)}{e(p^0, u^0)}
  \]

  The PPI understates inflation:
  \[
  \frac{p^1 \cdot x^1}{p^0 \cdot x^1}
  \leq \frac{e(p^1, u^1)}{e(p^0, u^1)}
  \]
\end{prop}

That is, the problem with both is that each has to fix consumption
bundles, so neither allows substitution that makes consumers weakly
better off in the EMP. Ideally we would just measure Hicksian
expenditure changes, but of course we can never do so.

\begin{definition}[Fisher Ideal Index]
  The Fisher Ideal Index is given by
  \[
  FII = \sqrt{ LPI \cdot PPI }
  \]
\end{definition}


\subsection{Choosing Functional Forms}
\label{sec:choos-funct-forms}


To generate simpler functional forms, we often take one of two
approaches:

\begin{enumerate}
\item Restrict preferences - impose that demand within group of goods
  is independent of prices outside the group. Thus the consumer is
  viewed as performing two step budgeting -- first allocate group
  expenditures and then within-group expenditures.
\item Restrict price movements - assume prices within group move
  proportionately. This is related to the problem set question about
  composite goods.
\end{enumerate}


\subsection{Integrability}
\label{sec:integrability}

If we observe a set of demand functions, a natural question would be
whether they are consistent with utility maximization. We already know
that rationality implies
\begin{enumerate}
\item Marshallian demand $x$ is homogenous of degree $0$
\item Marshallian demand $x$ satisfies Walras' law (i.e. $p \cdot
  x(p,w) = w$)
\item The Slutsky Matrix is symmetric and negative semidefinite
\end{enumerate}

\begin{prop}[Hurwicz-Uzawa]
  A set of continuously differentiable functions $x_i : X \times \R
  \to \R_+$ are the demand functions generated by some increasing,
  quasiconcave utility function $u$ if the satisfy Walras' law and
  have a symmetric negative semidefinite Slutsky substitution matrix.
\end{prop}

Basically, this says that we need two very simple properties on $x$ to
ensure that we are playing by the ``rules'' of micro -- i.e. that
utility analysis is appropriate.


\subsection{Demand Aggregation}
\label{sec:demand-aggregation}


\begin{definition}[Homothetic Preferences]
  Preferences are homothetic if $\forall x, y, \alpha > 0$,
  \[
  x \succeq y \implies \alpha x \succeq \alpha y
  \]
\end{definition}

\begin{prop}
  Homothetic preferences admit representation $u(x) = f(x)$ where
  $f(\alpha x) = \alpha f(x)$.
\end{prop}

\begin{prop}
  If preferences are homothetic, then demand is homogenous of degree
  $1$ in income
  \[
  x(p, \alpha w) = \alpha x(p,w)
  \]
  In general, then, if we let $\tilde x(p) := x(p, 1)$, then we can
  write
  \[
  x(p,w) = w \tilde x(p)
  \]

  Moreover, the income elasticity of demand is $1$, that is $\forall j,$
  \[
  \frac{\partial \log(x_j)}{\partial \log(w)} = 1
  \]
\end{prop}

\begin{theorem}[Chipman 1974]
  If individual preferences are homothetic but not necessarily
  identical, and incomes are proportional, then there exists a single
  preference ordering that generates aggregate demand.
\end{theorem}

\begin{prop}
  Aggregate demand is a function of aggregate wealth (i.e. typically
  aggregate demand is $X(p, w^1, \dots, w^N)$, and we want to know
  when we can represent it as $X(p, \sum w^i)$) only if 
  \[
  \frac{\partial x^k_i(p, w^k)}{\partial w^k}
  = \frac{\partial x^j_i(p, w^j)}{\partial w^j}
  \]
  $\forall$ goods $i$, and $\forall$ individuals $j,k$. The
  requirement is that wealth effects are the same across all
  individuals and across all wealth levels.
\end{prop}


In general, even if individuals have rational preferences that satisfy
GARP, if we aggregate agents and look only at aggregate demand, often
it will violate GARP. The easiest way to construct such an example is
to establish a satiation point for one or more of the agents.

Moreover, we generally cannot treat problems involving the
maximization over a set of agents' welfares instead as a problem
involving the maximization of the sum of their welfares. The issue
here is that the sum will depend upon the magnitudes of the individual
utilities, which themselves have arbitrary magnitudes because they can
be monotonically transformed and still represent preferences.

However, there is a very specifal case where we can do aggregation:
Gorman form. 

\subsubsection{Gorman Polar Form}
\label{sec:gorman-polar-form}


\begin{prop}[Gorman Polar Form]
  Demand is a function of aggregate wealth if and only if preferences
  represented by indirect utility functions of the form
  \[
  v^k(p, w^k) = a^k(p) + w^k \cdot b(p)
  \]
  where $k$ indexes the individual, hence $b(p)$ is uniform across
  consumers.
\end{prop}


In this case, aggregate demand is independent of the wealth
distribution, and maximizes:
\[
V(p,w) = \sum_k a^k(p) + w \cdot b(p)
\]
where $w = \sum_k w^k$ so we don't care about distribution. 

Now we can define a ``representative consumer'' who maximizes the
utilitarian social welfare function:
\[
W(u^1, \dots, u^k) = \sum_k u^k
\]
The intuition here is that there is some good with constant marginal
utility that is the same for all consumers. So regardless of wealth
each consumer will consume other goods until marginal utility drops
below $b(p)$ and then will put the remainder of wealth into the given
good. Then a representative consumer will consume each good until he
consumes the sum of individual demands, then puts the rest in the
constant marginal utility good.

And we can delve a bit more into the uniqueness proofs relating to
Gorman Form:

\begin{definition}[Quasi-Homothetic Preferences]
  Quasi-homothetic preferences give demands as
  \[
  x^k(p, w^k) = \alpha^k(p) + w^k \beta(p)
  \]
  where $\alpha^k$ is a person specific function but $\beta$ is
  common/identical across people.

  The aggregate demand for quasi-homothetic preferences itself
  quasi-homothetic, and is given by
  \[
  X(p,w) = \alpha(p) + w \cdot \beta(p)
  \]
  where $\alpha = \sum_k \alpha^k$ and $w = \sum_k w^k$.
\end{definition}


\begin{theorem}[Gorman 1961 Polar Form]
  The expenditure function is of the form
  \[
  e(p,u) = a(p) + u \cdot b(p)
  \]
  if and only if Marshallian demand is quasi-homothetic of form
  \[
  x(p,w) = \alpha(p) + w \cdot \beta(p)
  \]
  where $a(p)$ and $b(p)$ are homogenous of degree $1$ in price and
  \[
  \beta(p) = \frac{1}{b(p)}\frac{\partial b(p)}{\partial p}
  \]
  \[
  \alpha(p) = \frac{\partial a(p)}{\partial p} - \beta(p)a(p)
  \]

  Given the expenditure function, we can derive the indirect utility
  function of the form
  \[
  v(p,w) = \frac{w - a(p)}{b(p)}
  \]
\end{theorem}
