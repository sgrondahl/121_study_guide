
\begin{theorem}[Kuhn-Tucker Necessary Conditions]
  Suppose $\bar x \in \R^N$ is a local maximizer of the following that
  meets the given constraints:
  \begin{align}
    \max_{x \in \R^N} f(x) \quad s.t. \quad 
    & g_1(x) = b_1, \dots, g_m(x) = b_m & \text{(equality constraints)} \notag \\
    & h_1(x) \leq c_1, \dots, h_k(x) \leq c_k & \text{(inequality constraints)} \notag
  \end{align}
  then $\exists$ multipliers $\lambda_m \in \R$, $\lambda_k \in \R_+$
  such that
  \begin{enumerate}[(1)]
  \item $\forall n \in 1, \dots, N$,
    \[
    \frac{\partial f(\bar x)}{\partial x_n} 
    = \sum_{m=1}^M \lambda_m \frac{\partial g_m(\bar x)}{\partial x_n}
    + \sum_{k=1}^K \lambda_k \frac{\partial h_k(\bar x)}{\partial x_n}
    \]

  \item $\forall k \in 1, \dots, K$,
    \[
    \lambda_k(h_k(\bar x) - c_k) = 0
    \]
    (i.e. $\lambda_k = 0$ for any $h_k(\bar x) < c_k$).
  \end{enumerate}
\end{theorem}

\begin{prop}[Bang for the Buck]
  For a given utility function $u(x)$, prices that generate an
  interior solution $x^*$ must satisfy
  \[
  \frac{
    \frac{\partial u(x^*)}{\partial x_j}
  } {
    \frac{\partial u(x^*)}{\partial x_i}
  }
  = \frac{p_j}{p_i}
  \]
\end{prop}

\begin{proof}
  Suppose $x^*$ is an interior solution, and consider a feasible
  perturbation $x'$ where the agent spends $\delta$ less on $j$ and
  $\delta$ more on $i$. Then locally
  \[
  u(x') \approx u(x^*) 
  + \frac{\partial u(x^*)}{\partial x_j}\left(-\frac{\delta}{p_j}\right)
  + \frac{\partial u(x^*)}{\partial x_i}\left(\frac{\delta}{p_i}\right)
  \]
  and since $x^*$ is the interior maximimum, moving to $x'$ cannot
  increase utility, hence (for small $\delta$)
  \[
  u(x') \leq u(x^*) \implies
  \frac{\frac{\partial u(x^*)}{\partial x_j}}{p_j}
  \geq \frac{\frac{\partial u(x^*)}{\partial x_i}}{p_i}
  \]
  and vice versa if we spend $\delta$ mroe in $j$ and $\delta$ less on
  $i$. Thus we have two inequalitys which in combination give the
  desired result:
  \[
  \frac{
    \frac{\partial u(x^*)}{\partial x_j}
  } {
    \frac{\partial u(x^*)}{\partial x_i}
  }
  = \frac{p_j}{p_i}
  \]
\end{proof}


\begin{prop}[Envelope Theorem without Constraints]
  Let $\phi(\alpha) = \max_x f(x; \alpha)$ and let $x(\alpha)$ be the
  maximizing argument. Suppoes that $\alpha$ is differentiable at
  $\bar \alpha$ and that $\phi(\alpha)$ is uniquely maximized at $\bar
  \alpha$, then
  \[
  \frac{d \phi}{d \alpha_i} 
  = \frac{\partial f(x(\bar \alpha); \bar \alpha)}{\partial \alpha_i}
  \]
\end{prop}

\begin{proof}
  We get from the total derivative of a parametrized function (section
  \ref{sec:calculus}) that 
  \[
  \frac{d \phi(\alpha)}{d \alpha_i}
  = \frac{\partial f(x(\alpha); \alpha)}{\partial a_i}
  + \sum_j 
  \frac{\partial f(x(\alpha);\alpha)}{\partial x_j}
  \frac{\partial x_j(\alpha)}{\partial \alpha_i}
  \]
  but since $x(\bar \alpha)$ uniquely maximizes $f$,
  \[
  [Df(x(\bar \alpha); \bar \alpha)] = [0]
  \]
  so the first term in the sum is always zero, giving our desired result.
\end{proof}

\begin{theorem}[Envelope Theorem with Constraints]
  Suppose we want to keep track of some subset of parameters, $q \in
  \R^S$. The we can define
  \begin{align}
    v(q) := \max_{x \in \R^N} f(x; q) \quad s.t. \quad 
    & g_1(x) = b_1, \dots, g_m(x) = b_m & \text{(equality constraints)} \notag
  \end{align}
  with $x(q)$ the $\arg \max$ of the above. And let $\lambda_1, \dots,
  \lambda_m$ be the lagrange multipliers associated with the solution
  $v(q)$, then $\forall s = 1, \dots, S$,
  \[
  \frac{\partial v(q)}{\partial q_s} 
  = \frac{\partial f(x(q); q)}{\partial q_s}
  - \sum_{m=1}^M \lambda_m \frac{\partial g_m(x(q); q)}{\partial q_s}
  \]
\end{theorem}

\begin{proof}
  Again we get from the total derivative of a parametrized function (section
  \ref{sec:calculus}) that 
  \[
  \frac{d \phi(\alpha)}{d \alpha_i}
  = \frac{\partial f(x(\alpha); \alpha)}{\partial a_i}
  + \sum_j 
  \frac{\partial f(x(\alpha);\alpha)}{\partial x_j}
  \frac{\partial x_j(\alpha)}{\partial \alpha_i}
  \]
  but now because we have constraints, we don't necessarily have that
  the derivative of $f$ evaluated at $x(\bar \alpha)$ is zero. Rather,
  we have our KT multipliers:
  \[
  \frac{\partial f(x(\bar \alpha); \bar \alpha)}{\partial x_j}
  = \sum_{m=1}^M \lambda_m \frac{\partial g_m(x(\bar\alpha);\alpha)}{\partial x_j}
  \]
  which we can substitute into the above and switch the order of the
  sums to get
  \[
  \frac{d \phi(\alpha)}{d \alpha_i}
  = \frac{\partial f(x(\alpha); \alpha)}{\partial a_i}
  + \sum_{m=1}^M \lambda_m
  \left(
     \sum_j 
     \frac{\partial g_m(x(\bar\alpha);\alpha)}{\partial x_j}
     \frac{\partial x_j(\alpha)}{\partial \alpha_i}
  \right)
  \]
  because our constraints hold with equality, we have
  $g_m(x(\bar\alpha);\alpha) = b_m$, hence we can totally differentiate to get
  \[
  0 
  = \frac{\partial g_m(x(\bar\alpha);\alpha)}{\partial \alpha_i}
  + \sum_j 
  \frac{\partial g_m(x(\bar\alpha);\alpha)}{\partial x_j}
  \frac{dx_j}{d\alpha_i}
  \]
  which we can then substitute in to get
  \[
  \frac{\partial \phi(\bar \alpha)}{\partial \alpha_i}
  = \frac{\partial f(x(\bar\alpha);\alpha)}{\partial \alpha_i}
  - \sum_{m=1}^M\lambda_m 
  \frac{\partial g_m(x(\bar\alpha);\alpha)}{\partial \alpha_i}
  \]
\end{proof}

\begin{definition}[LHC]
  A correspondence $\Gamma:X \to Y$ is lower hemicontinuous (LHC) if
  $\forall x$ and $\forall y \in \Gamma(x)$, for any sequence $x_n \to
  x$ there exists a corresponding sequence $y_n \in \Gamma(x_n)$ with
  $y_n \to y$. (Any point $(x,y)$ in correspondence has sequence
  $(x_n, y_n)$ the converges to $(x,y)$). 
\end{definition}

\begin{definition}[UHC]
  A correspondence $\Gamma:X \to Y$ is upper hemicontinuous (UHC) if
  for every sequence $(x_n, y_n)$ in the correspondence, with $y_n \in
  \Gamma(x_n)$ such that $x_n \to x$, then $y_n \to y \in
  \Gamma(x)$. (Any sequence in the correspondence converges to a limit
  point in the correspondence).
\end{definition}

\begin{theorem}[Berge/Theorem of the Maximum]
  Let $X \subseteq \R^L$, $Y \subseteq \R^m$, $f : X \times Y \to \R$
  continuous, $\Gamma : X \to Y$ compact-valued and continuous
  correspondence. Then:
  \begin{enumerate}[(1)]
  \item $h(x) := \max_{y\in \Gamma(x)}f(x,y)$ is continuous
  \item $G(x) := \{y \in \Gamma(x): \; f(x,y) = h(x)\}$ is nonempty,
    compact-valued, and UHC
  \end{enumerate}
\end{theorem}
