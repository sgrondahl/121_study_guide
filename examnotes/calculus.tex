


\begin{definition}[Total Derivative]
  Consider the function $f(x(t), t)$, where one (or more) of the
  arguments varies in another argument. The total derivative of the
  function is given by
  \[
  \frac{df}{dt} 
  = \frac{\partial f}{\partial t} \frac{dt}{dt}
  + \frac{\partial f}{\partial x(t)} \frac{dx(t)}{dt}
  = \frac{\partial f}{\partial t} 
  + \frac{\partial f}{\partial x(t)} \frac{dx(t)}{dt}
  \]
\end{definition}

\begin{definition}[Chain Rule]
  In order to differentiate a composite function $f(g(x))$, the chain
  rule gives
  \[
  [D f \circ g(x)] = [Df(g(x))][Dg(x)]
  \]
  where the first term is the derivative of $f$ evaluated at $g(x)$
  and the second term is the derivative of $g$ evaluated at $x$.
\end{definition}

\begin{prop}[Total Derivative of Parametrized Function]
  Consider the function parametrized by $\alpha$, $f(x; \alpha)$, and
  let 
  \[
  \phi(\alpha) = \max_x f(x; \alpha)
  \]
  \[
  x(\alpha) = \arg \max_x f(x; \alpha)
  \]
  then the total derivative of $\phi$ with respect to each component
  of $\alpha$ is given by
  \[
  \frac{d \phi(\alpha)}{d \alpha_i}
  = \frac{\partial f(x(\alpha); \alpha)}{\partial a_i}
  + \sum_j 
  \frac{\partial f(x(\alpha);\alpha)}{\partial x_j}
  \frac{\partial x_j(\alpha)}{\partial \alpha_i}
  \]
\end{prop}

\begin{proof}
  This follows directly from applying the complete derivative to
  $\phi$ and then the chain rule when differentiating $x$.
\end{proof}
