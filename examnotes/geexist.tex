

\subsection{Existence of Walrasian Equilibrium}
\label{sec:exist-walr-equil}

We begin with a few preliminaries

\begin{theorem}[Brouwer Fixed Point]
  Suppoes $A \subset \R^L$ is nonempty, convex, compact, and $f: A \to
  A$ is continuous. Then $f$ has a fixed point:
  \[
  \exists x^* \in A \; : \; f(x^*) = x^*
  \]
\end{theorem}

\begin{theorem}[Kakutani Fixed Point]
  Suppose $A \subset \R^L$ is nonempty, compact, convex, and $f : A
  \to A$ is a non-empty, convex-valued, upper hemi-continuous
  correspondence. Then $f$ has a fixed point:
  \[
  \exists x^* \in A \; : \; x^* \in f(x^*)
  \]
\end{theorem}

\begin{proof}
  What we really want to do is to find a continuous function $g$ that
  falls inside of $f$ and then apply Brouwer, but unfortunately we
  can't do this in general. Instead, for each $n \in \N$, find a
  continuous function $g_n$ whose graph is everywhere within
  $\frac{1}{n}$ of $f$. 

  For each such function $g_n$, we can, by Brouwer, find fixed point
  $(a_n^*, a_n^*)$ in the graph of $g_n$. Thus $\exists (x_n, y_n)$ in
  the graph of $f$ such that
  \[
  |a_n^* - x_n| < \frac{1}{n} \quad \text{and} \quad |a_n^* - y_n| < \frac{1}{n}
  \]

  Since $A$ is compact, $\exists$ convergent subsequence $a^*_{n_k}
  \to a^*$ for some $a^* \in A$, hence both $x_{n_k}$ and $y_{n_k}$
  also converge to $a^*$. And since the sequence $(x_{n_k}, y_{n_k})$
  is in the graph of $f$ and $f$ is UHC it must be the case that the
  limit point $(a^*, a^*)$ is also in the graph of $f$, hence $a^* \in
  f(a^*)$.
\end{proof}


\begin{definition}[Excess Demand]
  The excess demand of consumer $i$ at prices $p$ is
  \[
  z^I(p) = x^i(p, p \cdot \omega^i) - \omega^i
  \]
\end{definition}

\begin{definition}[Aggregate Excess Demand]
  The aggregate excess demand at prices $p$ is
  \[
  z(p) = \sum_i z^i(p)
  \]
\end{definition}

\begin{prop}
  A price vector $p$ satisfies $z(p) = 0$ if and only if $(p,
  \{x^i\}_{i \in I})$ is a Walrasian equilibrium.
\end{prop}

Thus our goal to establish existsnce will be to show that a solution
to $z(p) = 0$ exists.

In order to simplify our study, we study prices on the simplex (since
only relative prices matter), i.e. prices sum to $1$. We denote these
prices on the simplex as $\Delta$ (when prices are nonnegative) or
$\Delta^0$ (when prices are strictly positive).


We continue to use the four assumptions from section
\ref{sec:four-assumptions}, but strengthen the monotonicity
condition. For simplicity, they are repeated below (with strong
monotonicity added). For each agent $i$,
\begin{enumerate}
\item $u^i$ is continuous
\item $u^i$ is strongly monotone in $x$ (i.e. incrementing
  \textit{any} element of $x$ increases $u$)
\item $u^i$ is concave
\item $\omega^i >> 0$
\end{enumerate}

The reason we add strong monotonicity is so consumers strictly value
increases in any single commodity, thus market clearing will be
exact. If we didn't add this, we could have all consumers value only a
single commodity which would permit many equilibria (because nobody
cares about the other commodities).

\begin{prop}
  In an exchange economy, if preferences satisfy our four assumptions
  and are strongly monotone, then $z(p)$ defined for $p \in \Delta^0$
  satisfies:
  \begin{enumerate}[(i)]
  \item $z$ is a continuoous function (we use this to establish that
    it must cross $0$ in the later proof of existence)
  \item homogeneity of degree $0$ (i.e. we can arbitrarily normalize
    prices)
  \item Walras' law: $p \cdot z(p) = 0 \; \forall p$
  \item bounded below: $z(p) \geq - \bar \omega$
  \item boundary condition: a sequence $p^n$ of prices that converges
    to a limit $p$ with some nonzero entries and at least one zero
    entry will have
    \[
    \lim_{n \to \infty} \max\{ z_1(p^n), \dots, z_L(p^n)\} = \infty
    \]
    that is, under the above prices, at least one consumer will have
    strictly positive wealth, and given the strongly monotone
    preferences, this consumer will consume unboundedly higher levels
    of at least one of the goods whose price is going to zero (if more
    than one are going to zero, the demands for all need not go to
    $+\infty$!)
  \end{enumerate}

\end{prop}

We use the proposition in the following way to prove existence (what
follows is just a rough sketch, though). For simplicity, we consider a
$2$ commodity market.
\begin{enumerate}
\item Homogeneity of degree $0$ allows us to normalize $p_2 = 1$ and
  consider only $p = p_1$.
\item Walras' Law allows us to check market clearning in $L-1$ markets
  (since the final market must then clear). Hence in our case we look
  only at the parket for good $1$ and consider solutions to the single
  equation $z_1(p, 1) = 0$.
\item We then try to establish two different prices $p$ such that
  $z_1(p, 1) > 0$ and $z_1(p, 1) < 0$:
  \begin{enumerate}
  \item First we let $p$ be very small. Walras Law gives that $p \cdot
    z_1(p, 1) + 1 \cdot z_2(p, 1) = 0$ and the boundary condition
    gives that $\max \{z_1(p,1), z_2(p,1)\} \to \infty$. Since $z_1$
    is bounded away from $-\infty$ by the boundary condition, it
    cannot be the case that $z_2 \to \infty$, hence we must have $z_1
    \to \infty$, i.e. we've estalished that at this tiny $p$, $z_1(p,
    1) > 0$
  \item Next we let $p$ be huge. We use h.o.d. zero to rewrite $z_1(p,
    1) = z_1(1, 1/p)$ so effectively we're letting $p_2$ become
    tiny. By the same argument as above (Walras) since $p_2$ is
    effectively tiny, then $z_2$ must be huge and the product is
    strictly postive, hence $p \cdot z_1$ must be strictly
    negative. But $p$ i s strictly positive, hence we know $z_1(p, 1)
    < 0$ for this very large $p$.
  \end{enumerate}
\item Finally, since we found prices that make $z_1 < 0$ and $z_1 >
  0$, since $z_1$ is by the proposition continuous, it must cross
  zero, i.e. $\exists p \; : \; z(p) = 0$.
\end{enumerate}

\begin{prop}
  If $z : \Delta \to \R^L$ is a continuous function that satisfies
  Walras' law, then $\exists p^* \in \Delta \; : \; z(p^*) \leq 0$.
\end{prop}

We now turn our attention to proving the existence of a Walrasian
equilibrium.

\begin{proof}
  The following is just a rough outline of how we would complete the
  proof.

  \begin{enumerate}
  \item Define a correspondence $f: \Delta^0 \to \Delta$ by
    \[
    f(p) 
    = \{q \in \Delta : q \cdot z(p) \geq q' \cdot z(p) \; \forall q' \in \Delta\}
    \]
    this correspondence identifies the goods with highest excess demand
  \item Extend domain of $f$ to $\Delta$ in order to make it UHC
  \item Verify that $p^* \in f(p^*) \implies p^* \in \Delta^0, \;
    z(p^*) = 0$, that is, any fixed point of the correspondence $f$
    should have strictly positive prices and zero excess demand
  \item Check that $f$ satisfies the hypotheses of Kakutani and apply
    it to conclude that $\exists p^* \in \Delta \; : \: p^* \in
    f(p^*)$, hence by the above $p^* \in \Delta^0$ and $z(p^*) = 0$
  \end{enumerate}
\end{proof}

\subsection{Large Economies}

Convexity of preferences has been central throughout our treatment of
GE. However, we might be able to get away with lack of convexity with
large numbers of agents. Suppose we have finitely many types of
agents, say $I$, and we have $r$ agents of each type. We can
approximate convexity with a fixed $I$ by blowing up $r$:

\begin{theorem}[Shapley-Folkman-Starr]
  (Very roughly): the sum of a large number of arbitrary sets in a
  finite dimensional vector space will be close to convex.
\end{theorem}


\subsection{General Facts about GE}
\label{sec:general-facts-ge}

Because $z$ is homogenous of degree $0$, we only care about relative
prices, hence we only need to check market clearing in $L-1$
markets. That is, the final market is fully determined by the other
$L-1$.

** TODO: what is the ``implication of homogeneity'' doing on p. 14 of
lecture 9? not clear where he gets the ``when $\alpha = 1$'' thing
from **


\begin{definition}[Regular Prices]
  An equilibrium price vector $p^*$ is regular if the Jacobian
  $Dz(p^*)$ has rank $L-1$. Whenever such an equilibrium is regular,
  we call the economy regular as well.
\end{definition}

If we have regularity, then we can trim off the final row and column
to get an $(L-1)\times(L-1)$ matrix that is full rank. We term this
$D\hat z(p)$. Moreover, if we fix $p^*_L = 1$ and let $q$ give the
\textit{relative} prices, and if we (for convenience) denote $\Omega =
\{\omega^i\}$, then we can define
\[
\hat z(q_1, \dots, q_{L-1}, \underbrace{1}_{p^*_L}; \Omega)
\]
which we can totally differentiate to get (in equilibrium, of course)
\[
D_q \hat z(p^*) \cdot dq + D_\Omega \hat z(p^*) \cdot d\Omega = 0
\]
which we can solve to get the comparative static result
\[
\frac{\partial q}{\partial \Omega} = - [D_q \hat z(p)]^{-1} [D_\Omega \hat z(p)]
\]
Hence regularity is the key condtion for comparative statics. It
allows us to invert the matrix and solve directly for comparative
statics results.

\begin{definition}[Locally Unique]
  An equilibrium price vector $p \in \Delta^0$ is locally unique if
  $\exists \epsilon > 0$ such that $\forall p' \in B_\epsilon(p),
  z(p') \neq 0$.
\end{definition}

\begin{theorem}
  Any regular equilibrium is locally unique.
\end{theorem}

\begin{proof}
  Apply the inverse function theorem, and if $p$ is not locally unique
  then $p$ is not regular (i.e. we can't invert the matrix, because if
  we could we would get a linear system around $p$, making it unique
  locally).
\end{proof}

\begin{theorem}
  A regular economy has a finite number of equilibria.
\end{theorem}

\begin{proof}
  By contradition: suppose not, then there is an infinite sequence of
  equilibrium price vectors in the simplex. Since the simplex is
  compact, we can find converging subsequence, and since z is
  continuous, the limit point of the sequence will not be locally
  unique (similar to the standard proof of Bolzano-Weierstrass
  theorem, where we make sequential cuts in the compact set; if there
  are infinitely many limit points, no matter how many cuts we make we
  have infinitely many points inside at least one tiny little area, so
  we don't have local uniqueness). Thus, p is not regular.
\end{proof}

\begin{theorem}[Debreu 1970]
  For almost every vector of initial endowments, the exchange economy
  is regular.
\end{theorem}

\begin{theorem}[Sonnenschein-Mantel-Debreu]
  Suppose $z$ is continuous, homogenous of degree $0$, and satisfies
  Walras' Law. Then for any $\epsilon > 0$, $\exists k$ consumers with
  continuous, strictly convex, and non-decreasing preferences such
  that $z$ is the aggregate excess demand for those $k$ consumers, for
  all $p$ such that $\frac{p_i}{||p||} > \epsilon \; \forall i$.
\end{theorem}

The above is the ``Anything Goes'' result -- any potential excess
demand function can be in fact a demand function. In Arrow's words
(1991), ``in the aggregate, the hypothesis of rational behavior has in
general no implications.'' The issue is that aggregation dissipates
the restrictions that rationality imposes (in the individual case,
these were very strong properties on the Slutsky matrix). 

The theorem says: give me an excess demand function $z(p)$ that
satsifies continuity, homogeneity of degree 0 and Walras’ law. Then I
can construct an economy (i.e. consumer preferences and endowments)
that generates this excess demand function. One interpretation is that
there are no testable implications of rationality (outside continuity,
homogeneity of degree 0, and Walras law) when looking at the agreggate
economy.  Why is this? \textit{The punchline is that this is all due
  to income effects. We do have a lot of testable implications for
  individual demand.}

However, if we impose additional functional form restrictions, we can
get something more tractible.
