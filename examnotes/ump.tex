
\begin{definition}[UMP]
  The consumer's utility maximization problem (UMP) is
  \[
  \max_{x\in B_{p,w}} u(x) \quad s.t. \quad p \cdot x \leq w
  \]
\end{definition}

\begin{definition}[Marshallian/Walrasian Demand]
  The Marshallian (Walrasian) Demand Correspondence $x(p,w) : X \times
  \R \to X$ is defined by
  \[
  x(p,w) = \{z \in B_{p,w} | u(z) = \max_{x \in B_{p,w}} u(x) \}
  \]
\end{definition}

\begin{prop}
  Suppose $u$ is continuous and satisfies local nonsatiation, then:
  \begin{enumerate}[(a)]
  \item The UMP problem has at least one solution
  \item $x(p,w)$ is homogenous of degree $0$, i.e. $x(\alpha p, \alpha
    w) = x(p,w), \; \forall \alpha > 0$
  \item $x(p,w)$ satisfies Walras' law, i.e. $p \cdot z = w, \;
    \forall z \in x(p,w)$
  \item If $u$ is strictly quasi-concave, then $x(p,w)$ is a function,
    i.e. each $x(p,w)$ contains a single bundle
  \end{enumerate}
\end{prop}

\begin{definition}[Elasticity of Demand]

  The elasticity of demand with income is 
  \[
  \eta_i = \frac{\partial x_i}{\partial w} \cdot \frac{w}{x_i}
  = \frac{\partial \log x_i}{\partial \log w}
  \]

  The elasticity of demand with price is 
  \[
  \epsilon_{ij}= \frac{\partial x_i}{\partial p_j} \cdot \frac{p_j}{x_i}
  = \frac{\partial \log x_i}{\partial \log p_j}
  \]
\end{definition}

\begin{prop}
  Let $s_i$ be the budget share of good $i$, $s_i = \frac{p_i
    x_i(p,w)}{w}$, then
  \begin{itemize}
  \item[(Engel aggregation)] $ \sum_{i=1}^n s_i \eta_i = 1$,
    i.e. total expenditure must change by an amount equal to any
    wealth change. This is equivalently stated as $\sum \frac{\partial
      x_i(p,w)}{\partial w} p_i = 1$, which comes from differentiating
    the budget constraint wrt $w$.
  \item[(Cournot aggregation)] $ \sum_{i=1}^n s_i \epsilon_{ij} =
    -s_j$, i.e. total expenditure cannot change in response to change
    in price. Equivalently, $\sum p_i \frac{\partial
      x_i(p,w)}{\partial p_j} = - x_j(p,w)$ which comes from
    differentiating the budget constraint wrt $p_j$.
  \end{itemize}
\end{prop}

\begin{definition}[Indirecty Utility]
  The indirect utility function $v(p,w): X \times R \to \R$ is defined by
  \[
  v(p,w) = \max_{x \in B_{p,w}} u(x)
  \]
  which in turn is equal to
  \[
  v(p,w) = u(x(p,w))
  \]
\end{definition}

\begin{prop}
  Suppose $u$ is continuous and satisfies local nonsatiation. Then the
  indirect utility function $v(p,w)$ is
  \begin{enumerate}[(a)]
  \item Homogenous of degree $0$, i.e. $v(\alpha p, \alpha w) = v(p,w)$
  \item Strictly increasing in $w$ and nonincreasing in $p_i, \forall i$
  \item Quasi-convex, i.e. $\{(p,w)|v(p,w) \leq v)\}$ is convex $\forall v$
  \item Continuous in $p$ and $w$
  \end{enumerate}
\end{prop}


\begin{prop}[Roy's Identity]
  Suppose that $u$ is continuous utility function and preferences are
  locally nonsatiated and strictly convex. Suppose that the indirect
  utility function $v(p, w)$ is differentiable at $(p^*, w^*) >>
  0$. Then $\forall i = 1, \dots, L$,
  \[
  x_i(p^*, w^*)
  = - \frac{
    \frac{\partial v(p^*, w^*)}{\partial p_i}
  } {
    \frac{\partial v(p^*, w^*)}{\partial w}
  }
  \]
\end{prop}

\begin{proof}
  This is a simple application of the Envelope Theorem. Recall the
  indirect utility function is given by
  \[
  v(p,w) = \max_x u(x) \quad s.t. \quad p \cdot x \leq w
  \]
  and since we only have one constraint that binds with equality, we
  have a single strictly positive multiplier $\lambda$. Let $x(p,w)$
  be the maximizing correspondence, and note that at the optimum
  \[
  \frac{\partial u(x(p,w))}{\partial p_i}
  = \frac{\partial u(x(p,w))}{\partial w}
  = 0
  \]
  so applying the Envelope Theorem we get
  \[
  \frac{\partial v(p,w)}{\partial p_i} = -\lambda x_i
  \]
  and
  \[
  \frac{\partial v(p,w)}{\partial w} = \lambda 
  \]
  so when we take the ratio we get the desired result.
\end{proof}

For examples, see 3.D.1 (p. 55 MWG).